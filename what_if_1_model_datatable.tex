\documentclass[11pt]{article}

    \usepackage[breakable]{tcolorbox}
    \usepackage{parskip} % Stop auto-indenting (to mimic markdown behaviour)
    

    % Basic figure setup, for now with no caption control since it's done
    % automatically by Pandoc (which extracts ![](path) syntax from Markdown).
    \usepackage{graphicx}
    % Maintain compatibility with old templates. Remove in nbconvert 6.0
    \let\Oldincludegraphics\includegraphics
    % Ensure that by default, figures have no caption (until we provide a
    % proper Figure object with a Caption API and a way to capture that
    % in the conversion process - todo).
    \usepackage{caption}
    \DeclareCaptionFormat{nocaption}{}
    \captionsetup{format=nocaption,aboveskip=0pt,belowskip=0pt}

    \usepackage{float}
    \floatplacement{figure}{H} % forces figures to be placed at the correct location
    \usepackage{xcolor} % Allow colors to be defined
    \usepackage{enumerate} % Needed for markdown enumerations to work
    \usepackage{geometry} % Used to adjust the document margins
    \usepackage{amsmath} % Equations
    \usepackage{amssymb} % Equations
    \usepackage{textcomp} % defines textquotesingle
    % Hack from http://tex.stackexchange.com/a/47451/13684:
    \AtBeginDocument{%
        \def\PYZsq{\textquotesingle}% Upright quotes in Pygmentized code
    }
    \usepackage{upquote} % Upright quotes for verbatim code
    \usepackage{eurosym} % defines \euro

    \usepackage{iftex}
    \ifPDFTeX
        \usepackage[T1]{fontenc}
        \IfFileExists{alphabeta.sty}{
              \usepackage{alphabeta}
          }{
              \usepackage[mathletters]{ucs}
              \usepackage[utf8x]{inputenc}
          }
    \else
        \usepackage{fontspec}
        \usepackage{unicode-math}
    \fi

    \usepackage{fancyvrb} % verbatim replacement that allows latex
    \usepackage{grffile} % extends the file name processing of package graphics
                         % to support a larger range
    \makeatletter % fix for old versions of grffile with XeLaTeX
    \@ifpackagelater{grffile}{2019/11/01}
    {
      % Do nothing on new versions
    }
    {
      \def\Gread@@xetex#1{%
        \IfFileExists{"\Gin@base".bb}%
        {\Gread@eps{\Gin@base.bb}}%
        {\Gread@@xetex@aux#1}%
      }
    }
    \makeatother
    \usepackage[Export]{adjustbox} % Used to constrain images to a maximum size
    \adjustboxset{max size={0.9\linewidth}{0.9\paperheight}}

    % The hyperref package gives us a pdf with properly built
    % internal navigation ('pdf bookmarks' for the table of contents,
    % internal cross-reference links, web links for URLs, etc.)
    \usepackage{hyperref}
    % The default LaTeX title has an obnoxious amount of whitespace. By default,
    % titling removes some of it. It also provides customization options.
    \usepackage{titling}
    \usepackage{longtable} % longtable support required by pandoc >1.10
    \usepackage{booktabs}  % table support for pandoc > 1.12.2
    \usepackage{array}     % table support for pandoc >= 2.11.3
    \usepackage{calc}      % table minipage width calculation for pandoc >= 2.11.1
    \usepackage[inline]{enumitem} % IRkernel/repr support (it uses the enumerate* environment)
    \usepackage[normalem]{ulem} % ulem is needed to support strikethroughs (\sout)
                                % normalem makes italics be italics, not underlines
    \usepackage{mathrsfs}
    

    
    % Colors for the hyperref package
    \definecolor{urlcolor}{rgb}{0,.145,.698}
    \definecolor{linkcolor}{rgb}{.71,0.21,0.01}
    \definecolor{citecolor}{rgb}{.12,.54,.11}

    % ANSI colors
    \definecolor{ansi-black}{HTML}{3E424D}
    \definecolor{ansi-black-intense}{HTML}{282C36}
    \definecolor{ansi-red}{HTML}{E75C58}
    \definecolor{ansi-red-intense}{HTML}{B22B31}
    \definecolor{ansi-green}{HTML}{00A250}
    \definecolor{ansi-green-intense}{HTML}{007427}
    \definecolor{ansi-yellow}{HTML}{DDB62B}
    \definecolor{ansi-yellow-intense}{HTML}{B27D12}
    \definecolor{ansi-blue}{HTML}{208FFB}
    \definecolor{ansi-blue-intense}{HTML}{0065CA}
    \definecolor{ansi-magenta}{HTML}{D160C4}
    \definecolor{ansi-magenta-intense}{HTML}{A03196}
    \definecolor{ansi-cyan}{HTML}{60C6C8}
    \definecolor{ansi-cyan-intense}{HTML}{258F8F}
    \definecolor{ansi-white}{HTML}{C5C1B4}
    \definecolor{ansi-white-intense}{HTML}{A1A6B2}
    \definecolor{ansi-default-inverse-fg}{HTML}{FFFFFF}
    \definecolor{ansi-default-inverse-bg}{HTML}{000000}

    % common color for the border for error outputs.
    \definecolor{outerrorbackground}{HTML}{FFDFDF}

    % commands and environments needed by pandoc snippets
    % extracted from the output of `pandoc -s`
    \providecommand{\tightlist}{%
      \setlength{\itemsep}{0pt}\setlength{\parskip}{0pt}}
    \DefineVerbatimEnvironment{Highlighting}{Verbatim}{commandchars=\\\{\}}
    % Add ',fontsize=\small' for more characters per line
    \newenvironment{Shaded}{}{}
    \newcommand{\KeywordTok}[1]{\textcolor[rgb]{0.00,0.44,0.13}{\textbf{{#1}}}}
    \newcommand{\DataTypeTok}[1]{\textcolor[rgb]{0.56,0.13,0.00}{{#1}}}
    \newcommand{\DecValTok}[1]{\textcolor[rgb]{0.25,0.63,0.44}{{#1}}}
    \newcommand{\BaseNTok}[1]{\textcolor[rgb]{0.25,0.63,0.44}{{#1}}}
    \newcommand{\FloatTok}[1]{\textcolor[rgb]{0.25,0.63,0.44}{{#1}}}
    \newcommand{\CharTok}[1]{\textcolor[rgb]{0.25,0.44,0.63}{{#1}}}
    \newcommand{\StringTok}[1]{\textcolor[rgb]{0.25,0.44,0.63}{{#1}}}
    \newcommand{\CommentTok}[1]{\textcolor[rgb]{0.38,0.63,0.69}{\textit{{#1}}}}
    \newcommand{\OtherTok}[1]{\textcolor[rgb]{0.00,0.44,0.13}{{#1}}}
    \newcommand{\AlertTok}[1]{\textcolor[rgb]{1.00,0.00,0.00}{\textbf{{#1}}}}
    \newcommand{\FunctionTok}[1]{\textcolor[rgb]{0.02,0.16,0.49}{{#1}}}
    \newcommand{\RegionMarkerTok}[1]{{#1}}
    \newcommand{\ErrorTok}[1]{\textcolor[rgb]{1.00,0.00,0.00}{\textbf{{#1}}}}
    \newcommand{\NormalTok}[1]{{#1}}

    % Additional commands for more recent versions of Pandoc
    \newcommand{\ConstantTok}[1]{\textcolor[rgb]{0.53,0.00,0.00}{{#1}}}
    \newcommand{\SpecialCharTok}[1]{\textcolor[rgb]{0.25,0.44,0.63}{{#1}}}
    \newcommand{\VerbatimStringTok}[1]{\textcolor[rgb]{0.25,0.44,0.63}{{#1}}}
    \newcommand{\SpecialStringTok}[1]{\textcolor[rgb]{0.73,0.40,0.53}{{#1}}}
    \newcommand{\ImportTok}[1]{{#1}}
    \newcommand{\DocumentationTok}[1]{\textcolor[rgb]{0.73,0.13,0.13}{\textit{{#1}}}}
    \newcommand{\AnnotationTok}[1]{\textcolor[rgb]{0.38,0.63,0.69}{\textbf{\textit{{#1}}}}}
    \newcommand{\CommentVarTok}[1]{\textcolor[rgb]{0.38,0.63,0.69}{\textbf{\textit{{#1}}}}}
    \newcommand{\VariableTok}[1]{\textcolor[rgb]{0.10,0.09,0.49}{{#1}}}
    \newcommand{\ControlFlowTok}[1]{\textcolor[rgb]{0.00,0.44,0.13}{\textbf{{#1}}}}
    \newcommand{\OperatorTok}[1]{\textcolor[rgb]{0.40,0.40,0.40}{{#1}}}
    \newcommand{\BuiltInTok}[1]{{#1}}
    \newcommand{\ExtensionTok}[1]{{#1}}
    \newcommand{\PreprocessorTok}[1]{\textcolor[rgb]{0.74,0.48,0.00}{{#1}}}
    \newcommand{\AttributeTok}[1]{\textcolor[rgb]{0.49,0.56,0.16}{{#1}}}
    \newcommand{\InformationTok}[1]{\textcolor[rgb]{0.38,0.63,0.69}{\textbf{\textit{{#1}}}}}
    \newcommand{\WarningTok}[1]{\textcolor[rgb]{0.38,0.63,0.69}{\textbf{\textit{{#1}}}}}


    % Define a nice break command that doesn't care if a line doesn't already
    % exist.
    \def\br{\hspace*{\fill} \\* }
    % Math Jax compatibility definitions
    \def\gt{>}
    \def\lt{<}
    \let\Oldtex\TeX
    \let\Oldlatex\LaTeX
    \renewcommand{\TeX}{\textrm{\Oldtex}}
    \renewcommand{\LaTeX}{\textrm{\Oldlatex}}
    % Document parameters
    % Document title
    \title{what\_if\_1\_model\_datatable}
    
    
    
    
    
% Pygments definitions
\makeatletter
\def\PY@reset{\let\PY@it=\relax \let\PY@bf=\relax%
    \let\PY@ul=\relax \let\PY@tc=\relax%
    \let\PY@bc=\relax \let\PY@ff=\relax}
\def\PY@tok#1{\csname PY@tok@#1\endcsname}
\def\PY@toks#1+{\ifx\relax#1\empty\else%
    \PY@tok{#1}\expandafter\PY@toks\fi}
\def\PY@do#1{\PY@bc{\PY@tc{\PY@ul{%
    \PY@it{\PY@bf{\PY@ff{#1}}}}}}}
\def\PY#1#2{\PY@reset\PY@toks#1+\relax+\PY@do{#2}}

\@namedef{PY@tok@w}{\def\PY@tc##1{\textcolor[rgb]{0.73,0.73,0.73}{##1}}}
\@namedef{PY@tok@c}{\let\PY@it=\textit\def\PY@tc##1{\textcolor[rgb]{0.24,0.48,0.48}{##1}}}
\@namedef{PY@tok@cp}{\def\PY@tc##1{\textcolor[rgb]{0.61,0.40,0.00}{##1}}}
\@namedef{PY@tok@k}{\let\PY@bf=\textbf\def\PY@tc##1{\textcolor[rgb]{0.00,0.50,0.00}{##1}}}
\@namedef{PY@tok@kp}{\def\PY@tc##1{\textcolor[rgb]{0.00,0.50,0.00}{##1}}}
\@namedef{PY@tok@kt}{\def\PY@tc##1{\textcolor[rgb]{0.69,0.00,0.25}{##1}}}
\@namedef{PY@tok@o}{\def\PY@tc##1{\textcolor[rgb]{0.40,0.40,0.40}{##1}}}
\@namedef{PY@tok@ow}{\let\PY@bf=\textbf\def\PY@tc##1{\textcolor[rgb]{0.67,0.13,1.00}{##1}}}
\@namedef{PY@tok@nb}{\def\PY@tc##1{\textcolor[rgb]{0.00,0.50,0.00}{##1}}}
\@namedef{PY@tok@nf}{\def\PY@tc##1{\textcolor[rgb]{0.00,0.00,1.00}{##1}}}
\@namedef{PY@tok@nc}{\let\PY@bf=\textbf\def\PY@tc##1{\textcolor[rgb]{0.00,0.00,1.00}{##1}}}
\@namedef{PY@tok@nn}{\let\PY@bf=\textbf\def\PY@tc##1{\textcolor[rgb]{0.00,0.00,1.00}{##1}}}
\@namedef{PY@tok@ne}{\let\PY@bf=\textbf\def\PY@tc##1{\textcolor[rgb]{0.80,0.25,0.22}{##1}}}
\@namedef{PY@tok@nv}{\def\PY@tc##1{\textcolor[rgb]{0.10,0.09,0.49}{##1}}}
\@namedef{PY@tok@no}{\def\PY@tc##1{\textcolor[rgb]{0.53,0.00,0.00}{##1}}}
\@namedef{PY@tok@nl}{\def\PY@tc##1{\textcolor[rgb]{0.46,0.46,0.00}{##1}}}
\@namedef{PY@tok@ni}{\let\PY@bf=\textbf\def\PY@tc##1{\textcolor[rgb]{0.44,0.44,0.44}{##1}}}
\@namedef{PY@tok@na}{\def\PY@tc##1{\textcolor[rgb]{0.41,0.47,0.13}{##1}}}
\@namedef{PY@tok@nt}{\let\PY@bf=\textbf\def\PY@tc##1{\textcolor[rgb]{0.00,0.50,0.00}{##1}}}
\@namedef{PY@tok@nd}{\def\PY@tc##1{\textcolor[rgb]{0.67,0.13,1.00}{##1}}}
\@namedef{PY@tok@s}{\def\PY@tc##1{\textcolor[rgb]{0.73,0.13,0.13}{##1}}}
\@namedef{PY@tok@sd}{\let\PY@it=\textit\def\PY@tc##1{\textcolor[rgb]{0.73,0.13,0.13}{##1}}}
\@namedef{PY@tok@si}{\let\PY@bf=\textbf\def\PY@tc##1{\textcolor[rgb]{0.64,0.35,0.47}{##1}}}
\@namedef{PY@tok@se}{\let\PY@bf=\textbf\def\PY@tc##1{\textcolor[rgb]{0.67,0.36,0.12}{##1}}}
\@namedef{PY@tok@sr}{\def\PY@tc##1{\textcolor[rgb]{0.64,0.35,0.47}{##1}}}
\@namedef{PY@tok@ss}{\def\PY@tc##1{\textcolor[rgb]{0.10,0.09,0.49}{##1}}}
\@namedef{PY@tok@sx}{\def\PY@tc##1{\textcolor[rgb]{0.00,0.50,0.00}{##1}}}
\@namedef{PY@tok@m}{\def\PY@tc##1{\textcolor[rgb]{0.40,0.40,0.40}{##1}}}
\@namedef{PY@tok@gh}{\let\PY@bf=\textbf\def\PY@tc##1{\textcolor[rgb]{0.00,0.00,0.50}{##1}}}
\@namedef{PY@tok@gu}{\let\PY@bf=\textbf\def\PY@tc##1{\textcolor[rgb]{0.50,0.00,0.50}{##1}}}
\@namedef{PY@tok@gd}{\def\PY@tc##1{\textcolor[rgb]{0.63,0.00,0.00}{##1}}}
\@namedef{PY@tok@gi}{\def\PY@tc##1{\textcolor[rgb]{0.00,0.52,0.00}{##1}}}
\@namedef{PY@tok@gr}{\def\PY@tc##1{\textcolor[rgb]{0.89,0.00,0.00}{##1}}}
\@namedef{PY@tok@ge}{\let\PY@it=\textit}
\@namedef{PY@tok@gs}{\let\PY@bf=\textbf}
\@namedef{PY@tok@gp}{\let\PY@bf=\textbf\def\PY@tc##1{\textcolor[rgb]{0.00,0.00,0.50}{##1}}}
\@namedef{PY@tok@go}{\def\PY@tc##1{\textcolor[rgb]{0.44,0.44,0.44}{##1}}}
\@namedef{PY@tok@gt}{\def\PY@tc##1{\textcolor[rgb]{0.00,0.27,0.87}{##1}}}
\@namedef{PY@tok@err}{\def\PY@bc##1{{\setlength{\fboxsep}{\string -\fboxrule}\fcolorbox[rgb]{1.00,0.00,0.00}{1,1,1}{\strut ##1}}}}
\@namedef{PY@tok@kc}{\let\PY@bf=\textbf\def\PY@tc##1{\textcolor[rgb]{0.00,0.50,0.00}{##1}}}
\@namedef{PY@tok@kd}{\let\PY@bf=\textbf\def\PY@tc##1{\textcolor[rgb]{0.00,0.50,0.00}{##1}}}
\@namedef{PY@tok@kn}{\let\PY@bf=\textbf\def\PY@tc##1{\textcolor[rgb]{0.00,0.50,0.00}{##1}}}
\@namedef{PY@tok@kr}{\let\PY@bf=\textbf\def\PY@tc##1{\textcolor[rgb]{0.00,0.50,0.00}{##1}}}
\@namedef{PY@tok@bp}{\def\PY@tc##1{\textcolor[rgb]{0.00,0.50,0.00}{##1}}}
\@namedef{PY@tok@fm}{\def\PY@tc##1{\textcolor[rgb]{0.00,0.00,1.00}{##1}}}
\@namedef{PY@tok@vc}{\def\PY@tc##1{\textcolor[rgb]{0.10,0.09,0.49}{##1}}}
\@namedef{PY@tok@vg}{\def\PY@tc##1{\textcolor[rgb]{0.10,0.09,0.49}{##1}}}
\@namedef{PY@tok@vi}{\def\PY@tc##1{\textcolor[rgb]{0.10,0.09,0.49}{##1}}}
\@namedef{PY@tok@vm}{\def\PY@tc##1{\textcolor[rgb]{0.10,0.09,0.49}{##1}}}
\@namedef{PY@tok@sa}{\def\PY@tc##1{\textcolor[rgb]{0.73,0.13,0.13}{##1}}}
\@namedef{PY@tok@sb}{\def\PY@tc##1{\textcolor[rgb]{0.73,0.13,0.13}{##1}}}
\@namedef{PY@tok@sc}{\def\PY@tc##1{\textcolor[rgb]{0.73,0.13,0.13}{##1}}}
\@namedef{PY@tok@dl}{\def\PY@tc##1{\textcolor[rgb]{0.73,0.13,0.13}{##1}}}
\@namedef{PY@tok@s2}{\def\PY@tc##1{\textcolor[rgb]{0.73,0.13,0.13}{##1}}}
\@namedef{PY@tok@sh}{\def\PY@tc##1{\textcolor[rgb]{0.73,0.13,0.13}{##1}}}
\@namedef{PY@tok@s1}{\def\PY@tc##1{\textcolor[rgb]{0.73,0.13,0.13}{##1}}}
\@namedef{PY@tok@mb}{\def\PY@tc##1{\textcolor[rgb]{0.40,0.40,0.40}{##1}}}
\@namedef{PY@tok@mf}{\def\PY@tc##1{\textcolor[rgb]{0.40,0.40,0.40}{##1}}}
\@namedef{PY@tok@mh}{\def\PY@tc##1{\textcolor[rgb]{0.40,0.40,0.40}{##1}}}
\@namedef{PY@tok@mi}{\def\PY@tc##1{\textcolor[rgb]{0.40,0.40,0.40}{##1}}}
\@namedef{PY@tok@il}{\def\PY@tc##1{\textcolor[rgb]{0.40,0.40,0.40}{##1}}}
\@namedef{PY@tok@mo}{\def\PY@tc##1{\textcolor[rgb]{0.40,0.40,0.40}{##1}}}
\@namedef{PY@tok@ch}{\let\PY@it=\textit\def\PY@tc##1{\textcolor[rgb]{0.24,0.48,0.48}{##1}}}
\@namedef{PY@tok@cm}{\let\PY@it=\textit\def\PY@tc##1{\textcolor[rgb]{0.24,0.48,0.48}{##1}}}
\@namedef{PY@tok@cpf}{\let\PY@it=\textit\def\PY@tc##1{\textcolor[rgb]{0.24,0.48,0.48}{##1}}}
\@namedef{PY@tok@c1}{\let\PY@it=\textit\def\PY@tc##1{\textcolor[rgb]{0.24,0.48,0.48}{##1}}}
\@namedef{PY@tok@cs}{\let\PY@it=\textit\def\PY@tc##1{\textcolor[rgb]{0.24,0.48,0.48}{##1}}}

\def\PYZbs{\char`\\}
\def\PYZus{\char`\_}
\def\PYZob{\char`\{}
\def\PYZcb{\char`\}}
\def\PYZca{\char`\^}
\def\PYZam{\char`\&}
\def\PYZlt{\char`\<}
\def\PYZgt{\char`\>}
\def\PYZsh{\char`\#}
\def\PYZpc{\char`\%}
\def\PYZdl{\char`\$}
\def\PYZhy{\char`\-}
\def\PYZsq{\char`\'}
\def\PYZdq{\char`\"}
\def\PYZti{\char`\~}
% for compatibility with earlier versions
\def\PYZat{@}
\def\PYZlb{[}
\def\PYZrb{]}
\makeatother


    % For linebreaks inside Verbatim environment from package fancyvrb.
    \makeatletter
        \newbox\Wrappedcontinuationbox
        \newbox\Wrappedvisiblespacebox
        \newcommand*\Wrappedvisiblespace {\textcolor{red}{\textvisiblespace}}
        \newcommand*\Wrappedcontinuationsymbol {\textcolor{red}{\llap{\tiny$\m@th\hookrightarrow$}}}
        \newcommand*\Wrappedcontinuationindent {3ex }
        \newcommand*\Wrappedafterbreak {\kern\Wrappedcontinuationindent\copy\Wrappedcontinuationbox}
        % Take advantage of the already applied Pygments mark-up to insert
        % potential linebreaks for TeX processing.
        %        {, <, #, %, $, ' and ": go to next line.
        %        _, }, ^, &, >, - and ~: stay at end of broken line.
        % Use of \textquotesingle for straight quote.
        \newcommand*\Wrappedbreaksatspecials {%
            \def\PYGZus{\discretionary{\char`\_}{\Wrappedafterbreak}{\char`\_}}%
            \def\PYGZob{\discretionary{}{\Wrappedafterbreak\char`\{}{\char`\{}}%
            \def\PYGZcb{\discretionary{\char`\}}{\Wrappedafterbreak}{\char`\}}}%
            \def\PYGZca{\discretionary{\char`\^}{\Wrappedafterbreak}{\char`\^}}%
            \def\PYGZam{\discretionary{\char`\&}{\Wrappedafterbreak}{\char`\&}}%
            \def\PYGZlt{\discretionary{}{\Wrappedafterbreak\char`\<}{\char`\<}}%
            \def\PYGZgt{\discretionary{\char`\>}{\Wrappedafterbreak}{\char`\>}}%
            \def\PYGZsh{\discretionary{}{\Wrappedafterbreak\char`\#}{\char`\#}}%
            \def\PYGZpc{\discretionary{}{\Wrappedafterbreak\char`\%}{\char`\%}}%
            \def\PYGZdl{\discretionary{}{\Wrappedafterbreak\char`\$}{\char`\$}}%
            \def\PYGZhy{\discretionary{\char`\-}{\Wrappedafterbreak}{\char`\-}}%
            \def\PYGZsq{\discretionary{}{\Wrappedafterbreak\textquotesingle}{\textquotesingle}}%
            \def\PYGZdq{\discretionary{}{\Wrappedafterbreak\char`\"}{\char`\"}}%
            \def\PYGZti{\discretionary{\char`\~}{\Wrappedafterbreak}{\char`\~}}%
        }
        % Some characters . , ; ? ! / are not pygmentized.
        % This macro makes them "active" and they will insert potential linebreaks
        \newcommand*\Wrappedbreaksatpunct {%
            \lccode`\~`\.\lowercase{\def~}{\discretionary{\hbox{\char`\.}}{\Wrappedafterbreak}{\hbox{\char`\.}}}%
            \lccode`\~`\,\lowercase{\def~}{\discretionary{\hbox{\char`\,}}{\Wrappedafterbreak}{\hbox{\char`\,}}}%
            \lccode`\~`\;\lowercase{\def~}{\discretionary{\hbox{\char`\;}}{\Wrappedafterbreak}{\hbox{\char`\;}}}%
            \lccode`\~`\:\lowercase{\def~}{\discretionary{\hbox{\char`\:}}{\Wrappedafterbreak}{\hbox{\char`\:}}}%
            \lccode`\~`\?\lowercase{\def~}{\discretionary{\hbox{\char`\?}}{\Wrappedafterbreak}{\hbox{\char`\?}}}%
            \lccode`\~`\!\lowercase{\def~}{\discretionary{\hbox{\char`\!}}{\Wrappedafterbreak}{\hbox{\char`\!}}}%
            \lccode`\~`\/\lowercase{\def~}{\discretionary{\hbox{\char`\/}}{\Wrappedafterbreak}{\hbox{\char`\/}}}%
            \catcode`\.\active
            \catcode`\,\active
            \catcode`\;\active
            \catcode`\:\active
            \catcode`\?\active
            \catcode`\!\active
            \catcode`\/\active
            \lccode`\~`\~
        }
    \makeatother

    \let\OriginalVerbatim=\Verbatim
    \makeatletter
    \renewcommand{\Verbatim}[1][1]{%
        %\parskip\z@skip
        \sbox\Wrappedcontinuationbox {\Wrappedcontinuationsymbol}%
        \sbox\Wrappedvisiblespacebox {\FV@SetupFont\Wrappedvisiblespace}%
        \def\FancyVerbFormatLine ##1{\hsize\linewidth
            \vtop{\raggedright\hyphenpenalty\z@\exhyphenpenalty\z@
                \doublehyphendemerits\z@\finalhyphendemerits\z@
                \strut ##1\strut}%
        }%
        % If the linebreak is at a space, the latter will be displayed as visible
        % space at end of first line, and a continuation symbol starts next line.
        % Stretch/shrink are however usually zero for typewriter font.
        \def\FV@Space {%
            \nobreak\hskip\z@ plus\fontdimen3\font minus\fontdimen4\font
            \discretionary{\copy\Wrappedvisiblespacebox}{\Wrappedafterbreak}
            {\kern\fontdimen2\font}%
        }%

        % Allow breaks at special characters using \PYG... macros.
        \Wrappedbreaksatspecials
        % Breaks at punctuation characters . , ; ? ! and / need catcode=\active
        \OriginalVerbatim[#1,codes*=\Wrappedbreaksatpunct]%
    }
    \makeatother

    % Exact colors from NB
    \definecolor{incolor}{HTML}{303F9F}
    \definecolor{outcolor}{HTML}{D84315}
    \definecolor{cellborder}{HTML}{CFCFCF}
    \definecolor{cellbackground}{HTML}{F7F7F7}

    % prompt
    \makeatletter
    \newcommand{\boxspacing}{\kern\kvtcb@left@rule\kern\kvtcb@boxsep}
    \makeatother
    \newcommand{\prompt}[4]{
        {\ttfamily\llap{{\color{#2}[#3]:\hspace{3pt}#4}}\vspace{-\baselineskip}}
    }
    

    
    % Prevent overflowing lines due to hard-to-break entities
    \sloppy
    % Setup hyperref package
    \hypersetup{
      breaklinks=true,  % so long urls are correctly broken across lines
      colorlinks=true,
      urlcolor=urlcolor,
      linkcolor=linkcolor,
      citecolor=citecolor,
      }
    % Slightly bigger margins than the latex defaults
    
    \geometry{verbose,tmargin=1in,bmargin=1in,lmargin=1in,rmargin=1in}
    
    

\begin{document}
    
    \maketitle
    
    

    
    \hypertarget{excel-what-if-analysis-with-python---part-1-models-and-data-tables}{%
\subsection{Excel ``What if?'' analysis with Python - Part 1: Models and
Data
Tables}\label{excel-what-if-analysis-with-python---part-1-models-and-data-tables}}

    Excel is widely used for building and using models of business problems
to explore the impact of various model inputs on key outputs. Built in
``what if?'' tools such as Excel
\href{https://support.microsoft.com/en-us/office/calculate-multiple-results-by-using-a-data-table-e95e2487-6ca6-4413-ad12-77542a5ea50b}{Data
Tables} and
\href{https://support.microsoft.com/en-us/office/use-goal-seek-to-find-the-result-you-want-by-adjusting-an-input-value-320cb99e-f4a4-417f-b1c3-4f369d6e66c7}{Goal
Seek} are well known to power spreadsheet modelers. How might we do
similar modeling and analysis using Python?

While Python has been gaining momentum in the business analytics world,
it is often used for data wrangling, analysis and visualization of
tablular data using tools like pandas and matplotlib or Seaborn. You can
find some great examples at \href{https://pbpython.com/}{Chris Moffit's
Practical Business Python blog}. I use Python all the time for such
tasks and teach a course called
\href{http://www.sba.oakland.edu/faculty/isken/courses/mis5470/}{Practical
Computing for Data Analytics} that is Python (and R) based. But, it got
me to thinking. What about those things for which Excel is well suited
such as building formula based models and doing sensitivity analysis on
these models? What would those look like in Python?

For example, here's a high level screenshot of a model that I assign for
homework in my
\href{http://www.sba.oakland.edu/faculty/isken/courses/mis5460/}{MIS
4460/5460 Business Analytics class} (a spreadsheet based modeling
class). It's a really simple model in which we are selling a single
product that we produce. There is a fixed cost to producing the product
as well as a variable production cost per unit. We can sell the product
for some price and we believe that demand for the product is related to
the selling price through a power function. Let's assume for now that we
have sufficient capacity to produce to demand and that all inputs are
deterministic (we'll deal with simulating uncertainty later in this
document).

    \begin{tcolorbox}[breakable, size=fbox, boxrule=1pt, pad at break*=1mm,colback=cellbackground, colframe=cellborder]
\prompt{In}{incolor}{ }{\boxspacing}
\begin{Verbatim}[commandchars=\\\{\}]
\PY{k+kn}{from} \PY{n+nn}{IPython}\PY{n+nn}{.}\PY{n+nn}{display} \PY{k+kn}{import} \PY{n}{Image}
\PY{n}{Image}\PY{p}{(}\PY{n}{filename}\PY{o}{=}\PY{l+s+s1}{\PYZsq{}}\PY{l+s+s1}{images/tech\PYZus{}sales\PYZus{}model.png}\PY{l+s+s1}{\PYZsq{}}\PY{p}{)}
\end{Verbatim}
\end{tcolorbox}

    The details aren't so important right now as is the overall structure of
the model. There's a few key inputs and some pretty straightforward
formulas for computing cost, revenue and profit. Notice the 1-way Data
Table being used to explore how profit varies for different selling
prices. There's a graph driven by the Data Table and some Text Boxes
used for annotation and summary interpretative comments. There's a
button that launches Goal Seek to find the break even selling price and
a 2-way Data Table (not shown) to explore the joint effect of selling
price and variable cost. Classic Excel modeling stuff. How might we go
about building a similar model using Python?

    What if we wanted to push it a little further and model some of the key
inputs with probability distributions to reflect our uncertainty about
their values? In the Excel world, we might use add-ins such as @Risk
which allow uncertain quantities to be directly modeled with probability
distributions. For example, we might have a key input such as the
exponent in the power function that relates selling price to demand that
is highly uncertain. By modeling it with a probability distribution and
then sampling from that distribution many times (essentially by
recalcing the spreadsheet) we can generate a bunch of possible values
for key outputs (e.g.~profit) and use statistics to summarize these
outputs using things like histograms and summary stats. Often this type
of simulation model is referred to as a \emph{Monte-Carlo} model to
suggest repeated sampling from one or more probability distributions
within an otherwise pretty static model. If you want to see such models
in action, check out my
\href{http://www.sba.oakland.edu/faculty/isken/courses/mis5460/simulation.html}{Simulation
Modeling with Excel page} from my Business Analytics course. Again, how
might we do this with Python?

    In the remainder of this notebook (and subsequent follow on notebooks),
we'll explore these questions and along the way introduce some basic
object-oriented (OO) programming concepts and other slightly more
advanced Python techniques. This is aimed at those who might have a
strong Excel based background but only a basic familiarity with Python
programming. We are going to use a slightly different problem example
than the one I've discussed above (we'll save that one for an ``exercise
for the motivated reader''). I've structured things as follows:

\textbf{Part 1 - Models and Data Tables: this notebook} * The bookstore
problem - Non-OO model for the bookstore problem - 1-, 2-, n-way Data
Tables with the non-OO model * Everything is an object in Python * An OO
model for the bookstore problem - 1-, 2-, n-way data tables with the OO
model

\textbf{Part 2 - Goal Seek: second notebook} * Goal Seek - a non-OO
hornet's nest - the OO approach

\textbf{Part 3 - Simulation: third notebook} * Monte-Carlo simulation

\textbf{Part 4 - Create package: fourth notebook} * share the results

\textbf{Part 5 - Simple GUI w/widgets: fifth notebook}

    \hypertarget{import-some-libraries}{%
\subsection{Import some libraries}\label{import-some-libraries}}

We'll need several Python libraries for this post.

    \begin{tcolorbox}[breakable, size=fbox, boxrule=1pt, pad at break*=1mm,colback=cellbackground, colframe=cellborder]
\prompt{In}{incolor}{1}{\boxspacing}
\begin{Verbatim}[commandchars=\\\{\}]
\PY{k+kn}{import} \PY{n+nn}{copy}

\PY{k+kn}{import} \PY{n+nn}{numpy} \PY{k}{as} \PY{n+nn}{np}
\PY{k+kn}{import} \PY{n+nn}{pandas} \PY{k}{as} \PY{n+nn}{pd}
\PY{k+kn}{import} \PY{n+nn}{matplotlib}\PY{n+nn}{.}\PY{n+nn}{pyplot} \PY{k}{as} \PY{n+nn}{plt}
\PY{k+kn}{from} \PY{n+nn}{mpl\PYZus{}toolkits}\PY{n+nn}{.}\PY{n+nn}{mplot3d} \PY{k+kn}{import} \PY{n}{Axes3D}
\PY{k+kn}{import} \PY{n+nn}{seaborn} \PY{k}{as} \PY{n+nn}{sns}
\end{Verbatim}
\end{tcolorbox}

    \begin{tcolorbox}[breakable, size=fbox, boxrule=1pt, pad at break*=1mm,colback=cellbackground, colframe=cellborder]
\prompt{In}{incolor}{2}{\boxspacing}
\begin{Verbatim}[commandchars=\\\{\}]
\PY{o}{\PYZpc{}}\PY{k}{matplotlib} inline
\end{Verbatim}
\end{tcolorbox}

    \hypertarget{bookstore-model}{%
\subsection{Bookstore model}\label{bookstore-model}}

This example is based on one in the
\href{https://host.kelley.iu.edu/albrightbooks/}{spreadsheet modeling
textbook(s) I've used in my classes since 2001}. I started out using
Practical Management Science by Winston and Albright and switched to
their Business Analytics: Data Analysis and Decision Making (Albright
and Winston) around 2013ish. In both books, they introduce the ``Walton
Bookstore'' problem in the chapter on Monte-Carlo simulation. Here's the
basic problem (with a few modifications):

\begin{itemize}
\tightlist
\item
  we have to place an order for a perishable product (e.g.~a calendar),
\item
  there's a known unit cost for each one ordered,
\item
  we have a known selling price,
\item
  demand is uncertain but we can model it with some simple probability
  distribution,
\item
  for each unsold item, we can get a partial refund of our unit cost,
\item
  we need to select the order quantity for our one order for the year;
  orders can only be in multiples of 25.
\end{itemize}

    \hypertarget{base-model---non-oo-approach}{%
\subsubsection{Base model - non-OO
approach}\label{base-model---non-oo-approach}}

Starting simple, let's create some initialized variables for the base
inputs.

    \begin{tcolorbox}[breakable, size=fbox, boxrule=1pt, pad at break*=1mm,colback=cellbackground, colframe=cellborder]
\prompt{In}{incolor}{3}{\boxspacing}
\begin{Verbatim}[commandchars=\\\{\}]
\PY{c+c1}{\PYZsh{} Base inputs}
\PY{n}{unit\PYZus{}cost} \PY{o}{=} \PY{l+m+mf}{7.50}
\PY{n}{selling\PYZus{}price} \PY{o}{=} \PY{l+m+mf}{10.00}
\PY{n}{unit\PYZus{}refund} \PY{o}{=} \PY{l+m+mf}{2.50}
\end{Verbatim}
\end{tcolorbox}

    Assume we've used historical data to estimate the mean and standard
deviation of demand. Furthermore, let's pretend that a histogram
revealed a relatively normal looking distribution. However, we are going
to start by:

\begin{itemize}
\tightlist
\item
  assuming away the uncertainty,
\item
  treat demand as deterministic using the mean,
\item
  we'll do some sensitivity analysis to the demand.
\end{itemize}

Later, we'll use Monte-Carlo simulation with normally distribution
demand.

    \begin{tcolorbox}[breakable, size=fbox, boxrule=1pt, pad at break*=1mm,colback=cellbackground, colframe=cellborder]
\prompt{In}{incolor}{4}{\boxspacing}
\begin{Verbatim}[commandchars=\\\{\}]
\PY{c+c1}{\PYZsh{} Demand parameters}
\PY{n}{demand\PYZus{}mean} \PY{o}{=} \PY{l+m+mi}{193}
\PY{n}{demand\PYZus{}sd} \PY{o}{=} \PY{l+m+mi}{40}

\PY{c+c1}{\PYZsh{} Deterministic model}
\PY{n}{demand} \PY{o}{=} \PY{n}{demand\PYZus{}mean}
\end{Verbatim}
\end{tcolorbox}

    Finally, let's set the initial order quantity. This will be the variable
we'll focus on in the sensitivity analysis (along with demand). Of
course, if we pretend that we know demand is really equal to 193, then
we'd only consider ordering 175 or 200 and would pick the one leading to
higher profit. Let's set it to 200.

    \begin{tcolorbox}[breakable, size=fbox, boxrule=1pt, pad at break*=1mm,colback=cellbackground, colframe=cellborder]
\prompt{In}{incolor}{5}{\boxspacing}
\begin{Verbatim}[commandchars=\\\{\}]
\PY{n}{order\PYZus{}quantity} \PY{o}{=} \PY{l+m+mi}{200}
\end{Verbatim}
\end{tcolorbox}

    Now we can compute the various cost and revenue components and get to
the bottom line profit.

\textbf{QUESTION 1:} Complete the lines of code below (answer at bottom
of notebook)

    \begin{tcolorbox}[breakable, size=fbox, boxrule=1pt, pad at break*=1mm,colback=cellbackground, colframe=cellborder]
\prompt{In}{incolor}{6}{\boxspacing}
\begin{Verbatim}[commandchars=\\\{\}]
\PY{n}{order\PYZus{}cost} \PY{o}{=} \PY{n}{unit\PYZus{}cost} \PY{o}{*} \PY{n}{order\PYZus{}quantity}
\PY{n}{sales\PYZus{}revenue} \PY{o}{=} \PY{err}{?}\PY{err}{?}\PY{err}{?} \PY{o}{*} \PY{n}{selling\PYZus{}price}
\PY{n}{refund\PYZus{}revenue} \PY{o}{=} \PY{err}{?}\PY{err}{?}\PY{err}{?} \PY{o}{*} \PY{n}{unit\PYZus{}refund}
\PY{n}{profit} \PY{o}{=} \PY{n}{sales\PYZus{}revenue} \PY{o}{+} \PY{n}{refund\PYZus{}revenue} \PY{o}{\PYZhy{}} \PY{n}{order\PYZus{}cost}
\end{Verbatim}
\end{tcolorbox}

    \begin{Verbatim}[commandchars=\\\{\}, frame=single, framerule=2mm, rulecolor=\color{outerrorbackground}]
\textcolor{ansi-cyan}{  Cell }\textcolor{ansi-green}{In[6], line 2}
\textcolor{ansi-red}{    sales\_revenue = ??? * selling\_price}
                    \^{}
\textcolor{ansi-red}{SyntaxError}\textcolor{ansi-red}{:} invalid syntax

    \end{Verbatim}

    \begin{tcolorbox}[breakable, size=fbox, boxrule=1pt, pad at break*=1mm,colback=cellbackground, colframe=cellborder]
\prompt{In}{incolor}{7}{\boxspacing}
\begin{Verbatim}[commandchars=\\\{\}]
\PY{n+nb}{print}\PY{p}{(}\PY{n}{profit}\PY{p}{)}
\end{Verbatim}
\end{tcolorbox}

    \begin{Verbatim}[commandchars=\\\{\}, frame=single, framerule=2mm, rulecolor=\color{outerrorbackground}]
\textcolor{ansi-red}{---------------------------------------------------------------------------}
\textcolor{ansi-red}{NameError}                                 Traceback (most recent call last)
Cell \textcolor{ansi-green}{In[7], line 1}
\textcolor{ansi-green}{----> 1} \def\tcRGB{\textcolor[RGB]}\expandafter\tcRGB\expandafter{\detokenize{0,135,0}}{print}(\setlength{\fboxsep}{0pt}\colorbox{ansi-yellow}{profit\strut})

\textcolor{ansi-red}{NameError}: name 'profit' is not defined
    \end{Verbatim}

    Of course, working in a Jupyter notebook is different than working in an
Excel workbook. If we modify one of the base input values, the value of
\texttt{profit} is \textbf{not} going to magically update.

    \begin{tcolorbox}[breakable, size=fbox, boxrule=1pt, pad at break*=1mm,colback=cellbackground, colframe=cellborder]
\prompt{In}{incolor}{ }{\boxspacing}
\begin{Verbatim}[commandchars=\\\{\}]
\PY{n}{unit\PYZus{}cost} \PY{o}{=} \PY{l+m+mf}{8.00}
\PY{n+nb}{print}\PY{p}{(}\PY{n}{profit}\PY{p}{)}
\end{Verbatim}
\end{tcolorbox}

    We either need to:

\begin{itemize}
\tightlist
\item
  rerun the code cell up above that computes the cost, revenue and
  profit,
\item
  or make the unit cost change up above when we first initialized and
  rerun all the cells,
\item
  or take a different approach\ldots{}
\end{itemize}

Since it feels like we are going to want to compute profit for different
combinations of base input values, it makes sense to create a function
to compute it. We'll do that eventually, but let's keep going with this
simplistic approach for a bit longer and learn some things about working
with vectors.

    \begin{tcolorbox}[breakable, size=fbox, boxrule=1pt, pad at break*=1mm,colback=cellbackground, colframe=cellborder]
\prompt{In}{incolor}{8}{\boxspacing}
\begin{Verbatim}[commandchars=\\\{\}]
\PY{c+c1}{\PYZsh{} Reset entire model}

\PY{c+c1}{\PYZsh{} Base inputs}
\PY{n}{unit\PYZus{}cost} \PY{o}{=} \PY{l+m+mf}{7.50}
\PY{n}{selling\PYZus{}price} \PY{o}{=} \PY{l+m+mf}{10.00}
\PY{n}{unit\PYZus{}refund} \PY{o}{=} \PY{l+m+mf}{2.50}
\PY{n}{demand} \PY{o}{=} \PY{l+m+mi}{193}
\PY{n}{order\PYZus{}quantity} \PY{o}{=} \PY{l+m+mi}{200}

\PY{c+c1}{\PYZsh{} Intermediate variables}
\PY{n}{order\PYZus{}cost} \PY{o}{=} \PY{n}{unit\PYZus{}cost} \PY{o}{*} \PY{n}{order\PYZus{}quantity}
\PY{n}{sales\PYZus{}revenue} \PY{o}{=} \PY{n+nb}{min}\PY{p}{(}\PY{n}{order\PYZus{}quantity}\PY{p}{,} \PY{n}{demand}\PY{p}{)} \PY{o}{*} \PY{n}{selling\PYZus{}price}
\PY{n}{refund\PYZus{}revenue} \PY{o}{=} \PY{n+nb}{max}\PY{p}{(}\PY{l+m+mi}{0}\PY{p}{,} \PY{n}{order\PYZus{}quantity} \PY{o}{\PYZhy{}} \PY{n}{demand}\PY{p}{)} \PY{o}{*} \PY{n}{unit\PYZus{}refund}

\PY{c+c1}{\PYZsh{} Key output variable}
\PY{n}{profit} \PY{o}{=} \PY{n}{sales\PYZus{}revenue} \PY{o}{+} \PY{n}{refund\PYZus{}revenue} \PY{o}{\PYZhy{}} \PY{n}{order\PYZus{}cost}

\PY{n}{message} \PY{o}{=} \PY{l+s+sa}{f}\PY{l+s+s2}{\PYZdq{}}\PY{l+s+si}{\PYZob{}}\PY{l+s+s1}{\PYZsq{}}\PY{l+s+s1}{Order Cost}\PY{l+s+s1}{\PYZsq{}}\PY{l+s+si}{:}\PY{l+s+s2}{15}\PY{l+s+si}{\PYZcb{}}\PY{l+s+si}{\PYZob{}}\PY{n}{order\PYZus{}cost}\PY{l+s+si}{:}\PY{l+s+s2}{10.2f}\PY{l+s+si}{\PYZcb{}}\PY{l+s+s2}{ }\PY{l+s+se}{\PYZbs{}n}\PY{l+s+s2}{\PYZdq{}} \PYZbs{}
          \PY{l+s+sa}{f}\PY{l+s+s2}{\PYZdq{}}\PY{l+s+si}{\PYZob{}}\PY{l+s+s1}{\PYZsq{}}\PY{l+s+s1}{Sales Revenue}\PY{l+s+s1}{\PYZsq{}}\PY{l+s+si}{:}\PY{l+s+s2}{15}\PY{l+s+si}{\PYZcb{}}\PY{l+s+si}{\PYZob{}}\PY{n}{sales\PYZus{}revenue}\PY{l+s+si}{:}\PY{l+s+s2}{10.2f}\PY{l+s+si}{\PYZcb{}}\PY{l+s+s2}{ }\PY{l+s+se}{\PYZbs{}n}\PY{l+s+s2}{\PYZdq{}} \PYZbs{}
          \PY{l+s+sa}{f}\PY{l+s+s2}{\PYZdq{}}\PY{l+s+si}{\PYZob{}}\PY{l+s+s1}{\PYZsq{}}\PY{l+s+s1}{Refund}\PY{l+s+s1}{\PYZsq{}}\PY{l+s+si}{:}\PY{l+s+s2}{15}\PY{l+s+si}{\PYZcb{}}\PY{l+s+si}{\PYZob{}}\PY{n}{refund\PYZus{}revenue}\PY{l+s+si}{:}\PY{l+s+s2}{10.2f}\PY{l+s+si}{\PYZcb{}}\PY{l+s+s2}{ }\PY{l+s+se}{\PYZbs{}n}\PY{l+s+s2}{\PYZdq{}} \PYZbs{}
          \PY{l+s+sa}{f}\PY{l+s+s2}{\PYZdq{}}\PY{l+s+si}{\PYZob{}}\PY{l+s+s1}{\PYZsq{}}\PY{l+s+s1}{Profit}\PY{l+s+s1}{\PYZsq{}}\PY{l+s+si}{:}\PY{l+s+s2}{15}\PY{l+s+si}{\PYZcb{}}\PY{l+s+si}{\PYZob{}}\PY{n}{profit}\PY{l+s+si}{:}\PY{l+s+s2}{10.2f}\PY{l+s+si}{\PYZcb{}}\PY{l+s+s2}{\PYZdq{}}

\PY{n+nb}{print}\PY{p}{(}\PY{n}{message}\PY{p}{)} 
\end{Verbatim}
\end{tcolorbox}

    \begin{Verbatim}[commandchars=\\\{\}]
Order Cost        1500.00
Sales Revenue     1930.00
Refund              17.50
Profit             447.50
    \end{Verbatim}

    Notice we've used Python's newish approach to formatting strings called
\href{https://realpython.com/python-f-strings/}{f-strings}. See
https://pyformat.info/ for the basics on the format specification (using
examples comparing the old \% style formatting versus the newer
\texttt{str.format()} approach) or the
\href{https://docs.python.org/3/library/string.html\#format-specification-mini-language}{official
docs for all the gory details}.

    \hypertarget{sensitivity-analysis-analogous-to-a-1-way-data-table}{%
\subsubsection{Sensitivity analysis analogous to a 1-way Data
Table}\label{sensitivity-analysis-analogous-to-a-1-way-data-table}}

To compute \texttt{profit} for several different values of
\texttt{order\_quantity} in Excel, we could do a 1-way Data Table.

In Python, we could set \texttt{order\_quantity} equal to a vector and
then recompute profit. Our model should spit out a vector of profit
values corresponding to the order quantities in our
\texttt{order\_quantity} vector. The numpy function
\href{https://numpy.org/doc/stable/reference/generated/numpy.arange.html}{arange}
can be used to create the vector.

    \begin{tcolorbox}[breakable, size=fbox, boxrule=1pt, pad at break*=1mm,colback=cellbackground, colframe=cellborder]
\prompt{In}{incolor}{9}{\boxspacing}
\begin{Verbatim}[commandchars=\\\{\}]
\PY{n}{order\PYZus{}quantity} \PY{o}{=} \PY{n}{np}\PY{o}{.}\PY{n}{arange}\PY{p}{(}\PY{n}{start}\PY{o}{=}\PY{l+m+mi}{50}\PY{p}{,} \PY{n}{stop}\PY{o}{=}\PY{l+m+mi}{301}\PY{p}{,} \PY{n}{step}\PY{o}{=}\PY{l+m+mi}{25}\PY{p}{)}
\PY{n+nb}{print}\PY{p}{(}\PY{n}{order\PYZus{}quantity}\PY{p}{)}
\end{Verbatim}
\end{tcolorbox}

    \begin{Verbatim}[commandchars=\\\{\}]
[ 50  75 100 125 150 175 200 225 250 275 300]
    \end{Verbatim}

    \begin{tcolorbox}[breakable, size=fbox, boxrule=1pt, pad at break*=1mm,colback=cellbackground, colframe=cellborder]
\prompt{In}{incolor}{10}{\boxspacing}
\begin{Verbatim}[commandchars=\\\{\}]
\PY{n+nb}{type}\PY{p}{(}\PY{n}{order\PYZus{}quantity}\PY{p}{)}
\end{Verbatim}
\end{tcolorbox}

            \begin{tcolorbox}[breakable, size=fbox, boxrule=.5pt, pad at break*=1mm, opacityfill=0]
\prompt{Out}{outcolor}{10}{\boxspacing}
\begin{Verbatim}[commandchars=\\\{\}]
numpy.ndarray
\end{Verbatim}
\end{tcolorbox}
        
    \textbf{QUESTION} Why do we set \texttt{stop=301} instead of 300? Also,
what's the data type of \texttt{order\_quantity}?

    \begin{tcolorbox}[breakable, size=fbox, boxrule=1pt, pad at break*=1mm,colback=cellbackground, colframe=cellborder]
\prompt{In}{incolor}{11}{\boxspacing}
\begin{Verbatim}[commandchars=\\\{\}]
\PY{n}{order\PYZus{}cost} \PY{o}{=} \PY{n}{unit\PYZus{}cost} \PY{o}{*} \PY{n}{order\PYZus{}quantity}
\PY{n}{sales\PYZus{}revenue} \PY{o}{=} \PY{n}{np}\PY{o}{.}\PY{n}{minimum}\PY{p}{(}\PY{n}{order\PYZus{}quantity}\PY{p}{,} \PY{n}{demand}\PY{p}{)} \PY{o}{*} \PY{n}{selling\PYZus{}price}
\PY{n}{refund\PYZus{}revenue} \PY{o}{=} \PY{n}{np}\PY{o}{.}\PY{n}{maximum}\PY{p}{(}\PY{l+m+mi}{0}\PY{p}{,} \PY{n}{order\PYZus{}quantity} \PY{o}{\PYZhy{}} \PY{n}{demand}\PY{p}{)} \PY{o}{*} \PY{n}{unit\PYZus{}refund}
\PY{n}{profit} \PY{o}{=} \PY{n}{sales\PYZus{}revenue} \PY{o}{+} \PY{n}{refund\PYZus{}revenue} \PY{o}{\PYZhy{}} \PY{n}{order\PYZus{}cost}
\PY{n+nb}{print}\PY{p}{(}\PY{n}{profit}\PY{p}{)}
\end{Verbatim}
\end{tcolorbox}

    \begin{Verbatim}[commandchars=\\\{\}]
[125.  187.5 250.  312.5 375.  437.5 447.5 322.5 197.5  72.5 -52.5]
    \end{Verbatim}

    \begin{tcolorbox}[breakable, size=fbox, boxrule=1pt, pad at break*=1mm,colback=cellbackground, colframe=cellborder]
\prompt{In}{incolor}{12}{\boxspacing}
\begin{Verbatim}[commandchars=\\\{\}]
\PY{n}{sales\PYZus{}revenue}
\end{Verbatim}
\end{tcolorbox}

            \begin{tcolorbox}[breakable, size=fbox, boxrule=.5pt, pad at break*=1mm, opacityfill=0]
\prompt{Out}{outcolor}{12}{\boxspacing}
\begin{Verbatim}[commandchars=\\\{\}]
array([ 500.,  750., 1000., 1250., 1500., 1750., 1930., 1930., 1930.,
       1930., 1930.])
\end{Verbatim}
\end{tcolorbox}
        
    We can't just use \texttt{min()}( with a scalar (\texttt{demand})
compared to a vector (\texttt{order\_quantity}). Need to use numpy
functions since they will do the necessary
\href{https://numpy.org/doc/stable/user/basics.broadcasting.html}{broadcasting}
of demand values to fill out a vector the same size as the
\texttt{order\_quantity} vector. In R, broadcasting is usually referred
to as \emph{recycling}.

    \begin{tcolorbox}[breakable, size=fbox, boxrule=1pt, pad at break*=1mm,colback=cellbackground, colframe=cellborder]
\prompt{In}{incolor}{13}{\boxspacing}
\begin{Verbatim}[commandchars=\\\{\}]
\PY{n}{np}\PY{o}{.}\PY{n}{minimum}\PY{p}{(}\PY{n}{order\PYZus{}quantity}\PY{p}{,} \PY{n}{demand}\PY{p}{)}
\end{Verbatim}
\end{tcolorbox}

            \begin{tcolorbox}[breakable, size=fbox, boxrule=.5pt, pad at break*=1mm, opacityfill=0]
\prompt{Out}{outcolor}{13}{\boxspacing}
\begin{Verbatim}[commandchars=\\\{\}]
array([ 50,  75, 100, 125, 150, 175, 193, 193, 193, 193, 193])
\end{Verbatim}
\end{tcolorbox}
        
    This won't work:

\begin{verbatim}
min(order_quantity, demand)

---------------------------------------------------------------------------
ValueError                                Traceback (most recent call last)
<ipython-input-42-675af9b521e5> in <module>
----> 1 min(order_quantity, demand)

ValueError: The truth value of an array with more than one element is ambiguous. Use a.any() or a.all()
\end{verbatim}

    Here's a simple plot to show how profit varies by order quantity.

    \begin{tcolorbox}[breakable, size=fbox, boxrule=1pt, pad at break*=1mm,colback=cellbackground, colframe=cellborder]
\prompt{In}{incolor}{14}{\boxspacing}
\begin{Verbatim}[commandchars=\\\{\}]
\PY{n}{plt}\PY{o}{.}\PY{n}{plot}\PY{p}{(}\PY{n}{order\PYZus{}quantity}\PY{p}{,} \PY{n}{profit}\PY{p}{,} \PY{n}{label}\PY{o}{=}\PY{l+s+s2}{\PYZdq{}}\PY{l+s+s2}{Profit}\PY{l+s+s2}{\PYZdq{}}\PY{p}{)}
\PY{n}{plt}\PY{o}{.}\PY{n}{xlabel}\PY{p}{(}\PY{l+s+s1}{\PYZsq{}}\PY{l+s+s1}{Order Quantity}\PY{l+s+s1}{\PYZsq{}}\PY{p}{)}
\PY{n}{plt}\PY{o}{.}\PY{n}{ylabel}\PY{p}{(}\PY{l+s+s1}{\PYZsq{}}\PY{l+s+s1}{Profit}\PY{l+s+s1}{\PYZsq{}}\PY{p}{)}
\PY{n}{plt}\PY{o}{.}\PY{n}{legend}\PY{p}{(}\PY{p}{)}
\PY{n}{plt}\PY{o}{.}\PY{n}{show}\PY{p}{(}\PY{p}{)}\PY{p}{;}
\end{Verbatim}
\end{tcolorbox}

    \begin{center}
    \adjustimage{max size={0.9\linewidth}{0.9\paperheight}}{output_34_0.png}
    \end{center}
    { \hspace*{\fill} \\}
    
    \textbf{QUESTION} What about a 2-way Data Table? Why can't we just set
one of the other inputs (e.g.~demand) to a vector and do the same thing
we did above?

    One approach to creating a 2-way (actually, n-way) is to create a profit
function that we can call from within a list comprehension.

    \begin{tcolorbox}[breakable, size=fbox, boxrule=1pt, pad at break*=1mm,colback=cellbackground, colframe=cellborder]
\prompt{In}{incolor}{15}{\boxspacing}
\begin{Verbatim}[commandchars=\\\{\}]
\PY{k}{def} \PY{n+nf}{bookstore\PYZus{}profit}\PY{p}{(}\PY{n}{unit\PYZus{}cost}\PY{p}{,} \PY{n}{selling\PYZus{}price}\PY{p}{,} \PY{n}{unit\PYZus{}refund}\PY{p}{,} \PY{n}{order\PYZus{}quantity}\PY{p}{,} \PY{n}{demand}\PY{p}{)}\PY{p}{:}
    \PY{l+s+sd}{\PYZsq{}\PYZsq{}\PYZsq{}}
\PY{l+s+sd}{    Compute profit in bookstore model}
\PY{l+s+sd}{    \PYZsq{}\PYZsq{}\PYZsq{}}
    \PY{n}{order\PYZus{}cost} \PY{o}{=} \PY{n}{unit\PYZus{}cost} \PY{o}{*} \PY{n}{order\PYZus{}quantity}
    \PY{n}{sales\PYZus{}revenue} \PY{o}{=} \PY{n}{np}\PY{o}{.}\PY{n}{minimum}\PY{p}{(}\PY{n}{order\PYZus{}quantity}\PY{p}{,} \PY{n}{demand}\PY{p}{)} \PY{o}{*} \PY{n}{selling\PYZus{}price}
    \PY{n}{refund\PYZus{}revenue} \PY{o}{=} \PY{n}{np}\PY{o}{.}\PY{n}{maximum}\PY{p}{(}\PY{l+m+mi}{0}\PY{p}{,} \PY{n}{order\PYZus{}quantity} \PY{o}{\PYZhy{}} \PY{n}{demand}\PY{p}{)}
    \PY{n}{profit} \PY{o}{=} \PY{n}{sales\PYZus{}revenue} \PY{o}{+} \PY{n}{refund\PYZus{}revenue} \PY{o}{\PYZhy{}} \PY{n}{order\PYZus{}cost}
    \PY{k}{return} \PY{n}{profit}
\end{Verbatim}
\end{tcolorbox}

    A
\href{https://www.pythontutorial.net/python-basics/python-list-comprehensions/}{list
comprehension} can be used to create an n-way Data Table. List
comprehensions provide a nice syntactical shortcut to creating a list
based one looping over one or more iterables and computing associated
values to store in the list. As a simple first example that's related to
the model we are working on, let's compute the difference between demand
and order quantity for all the combinations of those variables in the
following ranges:

    \begin{tcolorbox}[breakable, size=fbox, boxrule=1pt, pad at break*=1mm,colback=cellbackground, colframe=cellborder]
\prompt{In}{incolor}{16}{\boxspacing}
\begin{Verbatim}[commandchars=\\\{\}]
\PY{n}{demand\PYZus{}range} \PY{o}{=} \PY{n}{np}\PY{o}{.}\PY{n}{arange}\PY{p}{(}\PY{l+m+mi}{50}\PY{p}{,} \PY{l+m+mi}{101}\PY{p}{,} \PY{l+m+mi}{10}\PY{p}{)}
\PY{n}{order\PYZus{}quantity\PYZus{}range} \PY{o}{=} \PY{n}{np}\PY{o}{.}\PY{n}{arange}\PY{p}{(}\PY{l+m+mi}{50}\PY{p}{,} \PY{l+m+mi}{101}\PY{p}{,} \PY{l+m+mi}{25}\PY{p}{)}
\PY{n+nb}{print}\PY{p}{(}\PY{n}{demand\PYZus{}range}\PY{p}{)}
\PY{n+nb}{print}\PY{p}{(}\PY{n}{order\PYZus{}quantity\PYZus{}range}\PY{p}{)}
\end{Verbatim}
\end{tcolorbox}

    \begin{Verbatim}[commandchars=\\\{\}]
[ 50  60  70  80  90 100]
[ 50  75 100]
    \end{Verbatim}

    Let's first use a list comprehension to list the combinations (we'll
store each combo in a tuple). Each combination is a \emph{scenario} for
which we'd like to compute profit.

    \begin{tcolorbox}[breakable, size=fbox, boxrule=1pt, pad at break*=1mm,colback=cellbackground, colframe=cellborder]
\prompt{In}{incolor}{17}{\boxspacing}
\begin{Verbatim}[commandchars=\\\{\}]
\PY{n+nb}{print}\PY{p}{(}\PY{p}{[}\PY{p}{(}\PY{n}{d}\PY{p}{,} \PY{n}{o}\PY{p}{)} \PY{k}{for} \PY{n}{d} \PY{o+ow}{in} \PY{n}{demand\PYZus{}range} \PY{k}{for} \PY{n}{o} \PY{o+ow}{in} \PY{n}{order\PYZus{}quantity\PYZus{}range}\PY{p}{]}\PY{p}{)}
\end{Verbatim}
\end{tcolorbox}

    \begin{Verbatim}[commandchars=\\\{\}]
[(50, 50), (50, 75), (50, 100), (60, 50), (60, 75), (60, 100), (70, 50), (70,
75), (70, 100), (80, 50), (80, 75), (80, 100), (90, 50), (90, 75), (90, 100),
(100, 50), (100, 75), (100, 100)]
    \end{Verbatim}

    This time I'll compute a list of differences between demand and order
quantity.

    \begin{tcolorbox}[breakable, size=fbox, boxrule=1pt, pad at break*=1mm,colback=cellbackground, colframe=cellborder]
\prompt{In}{incolor}{18}{\boxspacing}
\begin{Verbatim}[commandchars=\\\{\}]
\PY{n+nb}{print}\PY{p}{(}\PY{p}{[}\PY{n}{d} \PY{o}{\PYZhy{}} \PY{n}{o} \PY{k}{for} \PY{n}{d} \PY{o+ow}{in} \PY{n}{demand\PYZus{}range} \PY{k}{for} \PY{n}{o} \PY{o+ow}{in} \PY{n}{order\PYZus{}quantity\PYZus{}range}\PY{p}{]}\PY{p}{)}
\end{Verbatim}
\end{tcolorbox}

    \begin{Verbatim}[commandchars=\\\{\}]
[0, -25, -50, 10, -15, -40, 20, -5, -30, 30, 5, -20, 40, 15, -10, 50, 25, 0]
    \end{Verbatim}

    Here's a 2-way with demand and order quantity - the other inputs are
held constant at their current values and the \texttt{bookstore\_profit}
function is used to compute profit. The list comprehension will generate
a list of tuples, \texttt{(demand,\ order\ quantity,\ profit)}. To
facilitate further analysis, we'll create a pandas DataFrame from this
list.

    \begin{tcolorbox}[breakable, size=fbox, boxrule=1pt, pad at break*=1mm,colback=cellbackground, colframe=cellborder]
\prompt{In}{incolor}{19}{\boxspacing}
\begin{Verbatim}[commandchars=\\\{\}]
\PY{n}{demand\PYZus{}range} \PY{o}{=} \PY{n}{np}\PY{o}{.}\PY{n}{arange}\PY{p}{(}\PY{l+m+mi}{50}\PY{p}{,} \PY{l+m+mi}{301}\PY{p}{,} \PY{l+m+mi}{5}\PY{p}{)}
\PY{n}{order\PYZus{}quantity\PYZus{}range} \PY{o}{=} \PY{n}{np}\PY{o}{.}\PY{n}{arange}\PY{p}{(}\PY{l+m+mi}{50}\PY{p}{,} \PY{l+m+mi}{301}\PY{p}{,} \PY{l+m+mi}{25}\PY{p}{)}
\end{Verbatim}
\end{tcolorbox}

    \begin{tcolorbox}[breakable, size=fbox, boxrule=1pt, pad at break*=1mm,colback=cellbackground, colframe=cellborder]
\prompt{In}{incolor}{20}{\boxspacing}
\begin{Verbatim}[commandchars=\\\{\}]
\PY{c+c1}{\PYZsh{} Create data table (as a list of tuples)}
\PY{n}{data\PYZus{}table\PYZus{}1} \PY{o}{=} \PY{p}{[}\PY{p}{(}\PY{n}{d}\PY{p}{,} \PY{n}{oq}\PY{p}{,} \PY{n}{bookstore\PYZus{}profit}\PY{p}{(}\PY{n}{unit\PYZus{}cost}\PY{p}{,} \PY{n}{selling\PYZus{}price}\PY{p}{,} \PY{n}{unit\PYZus{}refund}\PY{p}{,} \PY{n}{oq}\PY{p}{,} \PY{n}{d}\PY{p}{)}\PY{p}{)} 
                 \PY{k}{for} \PY{n}{d} \PY{o+ow}{in} \PY{n}{demand\PYZus{}range} \PY{k}{for} \PY{n}{oq} \PY{o+ow}{in} \PY{n}{order\PYZus{}quantity\PYZus{}range}\PY{p}{]}

\PY{c+c1}{\PYZsh{} Convert to dataframe}
\PY{n}{dtbl\PYZus{}1\PYZus{}df} \PY{o}{=} \PY{n}{pd}\PY{o}{.}\PY{n}{DataFrame}\PY{p}{(}\PY{n}{data\PYZus{}table\PYZus{}1}\PY{p}{,} \PY{n}{columns}\PY{o}{=}\PY{p}{[}\PY{l+s+s1}{\PYZsq{}}\PY{l+s+s1}{Demand}\PY{l+s+s1}{\PYZsq{}}\PY{p}{,} \PY{l+s+s1}{\PYZsq{}}\PY{l+s+s1}{OrderQuantity}\PY{l+s+s1}{\PYZsq{}}\PY{p}{,} \PY{l+s+s1}{\PYZsq{}}\PY{l+s+s1}{Profit}\PY{l+s+s1}{\PYZsq{}}\PY{p}{]}\PY{p}{)}
\PY{n}{dtbl\PYZus{}1\PYZus{}df}\PY{o}{.}\PY{n}{head}\PY{p}{(}\PY{l+m+mi}{25}\PY{p}{)}
\end{Verbatim}
\end{tcolorbox}

            \begin{tcolorbox}[breakable, size=fbox, boxrule=.5pt, pad at break*=1mm, opacityfill=0]
\prompt{Out}{outcolor}{20}{\boxspacing}
\begin{Verbatim}[commandchars=\\\{\}]
    Demand  OrderQuantity  Profit
0       50             50   125.0
1       50             75   -37.5
2       50            100  -200.0
3       50            125  -362.5
4       50            150  -525.0
5       50            175  -687.5
6       50            200  -850.0
7       50            225 -1012.5
8       50            250 -1175.0
9       50            275 -1337.5
10      50            300 -1500.0
11      55             50   125.0
12      55             75     7.5
13      55            100  -155.0
14      55            125  -317.5
15      55            150  -480.0
16      55            175  -642.5
17      55            200  -805.0
18      55            225  -967.5
19      55            250 -1130.0
20      55            275 -1292.5
21      55            300 -1455.0
22      60             50   125.0
23      60             75    52.5
24      60            100  -110.0
\end{Verbatim}
\end{tcolorbox}
        
    Here's a quick scatter plot showing how profit (mapped to color hue)
varies jointly by order quantity and demand. No surprises.

    \begin{tcolorbox}[breakable, size=fbox, boxrule=1pt, pad at break*=1mm,colback=cellbackground, colframe=cellborder]
\prompt{In}{incolor}{21}{\boxspacing}
\begin{Verbatim}[commandchars=\\\{\}]
\PY{n}{sns}\PY{o}{.}\PY{n}{set\PYZus{}style}\PY{p}{(}\PY{l+s+s2}{\PYZdq{}}\PY{l+s+s2}{darkgrid}\PY{l+s+s2}{\PYZdq{}}\PY{p}{)}
\PY{n}{sns}\PY{o}{.}\PY{n}{scatterplot}\PY{p}{(}\PY{n}{x}\PY{o}{=}\PY{l+s+s2}{\PYZdq{}}\PY{l+s+s2}{Demand}\PY{l+s+s2}{\PYZdq{}}\PY{p}{,} \PY{n}{y}\PY{o}{=}\PY{l+s+s2}{\PYZdq{}}\PY{l+s+s2}{OrderQuantity}\PY{l+s+s2}{\PYZdq{}}\PY{p}{,} \PY{n}{hue}\PY{o}{=}\PY{l+s+s2}{\PYZdq{}}\PY{l+s+s2}{Profit}\PY{l+s+s2}{\PYZdq{}}\PY{p}{,} \PY{n}{data}\PY{o}{=}\PY{n}{dtbl\PYZus{}1\PYZus{}df}\PY{p}{,} \PY{n}{palette}\PY{o}{=}\PY{l+s+s2}{\PYZdq{}}\PY{l+s+s2}{viridis}\PY{l+s+s2}{\PYZdq{}}\PY{p}{)}\PY{p}{;}
\end{Verbatim}
\end{tcolorbox}

    \begin{center}
    \adjustimage{max size={0.9\linewidth}{0.9\paperheight}}{output_48_0.png}
    \end{center}
    { \hspace*{\fill} \\}
    
    For fun, let's try a 3d plot.

https://matplotlib.org/mpl\_toolkits/mplot3d/tutorial.html\#mplot3d-tutorial

    \begin{tcolorbox}[breakable, size=fbox, boxrule=1pt, pad at break*=1mm,colback=cellbackground, colframe=cellborder]
\prompt{In}{incolor}{23}{\boxspacing}
\begin{Verbatim}[commandchars=\\\{\}]
\PY{c+c1}{\PYZsh{} Make the plot }
\PY{n}{fig} \PY{o}{=} \PY{n}{plt}\PY{o}{.}\PY{n}{figure}\PY{p}{(}\PY{p}{)}
\PY{n}{ax} \PY{o}{=} \PY{n}{fig}\PY{o}{.}\PY{n}{add\PYZus{}subplot}\PY{p}{(}\PY{l+m+mi}{111}\PY{p}{,} \PY{n}{projection}\PY{o}{=}\PY{l+s+s1}{\PYZsq{}}\PY{l+s+s1}{3d}\PY{l+s+s1}{\PYZsq{}}\PY{p}{)}
\PY{n}{ax}\PY{o}{.}\PY{n}{scatter}\PY{p}{(}\PY{n}{dtbl\PYZus{}1\PYZus{}df}\PY{p}{[}\PY{l+s+s1}{\PYZsq{}}\PY{l+s+s1}{OrderQuantity}\PY{l+s+s1}{\PYZsq{}}\PY{p}{]}\PY{p}{,} \PY{n}{dtbl\PYZus{}1\PYZus{}df}\PY{p}{[}\PY{l+s+s1}{\PYZsq{}}\PY{l+s+s1}{Demand}\PY{l+s+s1}{\PYZsq{}}\PY{p}{]}\PY{p}{,} \PY{n}{dtbl\PYZus{}1\PYZus{}df}\PY{p}{[}\PY{l+s+s1}{\PYZsq{}}\PY{l+s+s1}{Profit}\PY{l+s+s1}{\PYZsq{}}\PY{p}{]}\PY{p}{,} 
                \PY{n}{cmap}\PY{o}{=}\PY{n}{plt}\PY{o}{.}\PY{n}{cm}\PY{o}{.}\PY{n}{viridis}\PY{p}{,} \PY{n}{c}\PY{o}{=}\PY{n}{dtbl\PYZus{}1\PYZus{}df}\PY{p}{[}\PY{l+s+s1}{\PYZsq{}}\PY{l+s+s1}{Profit}\PY{l+s+s1}{\PYZsq{}}\PY{p}{]}\PY{p}{)}

\PY{n}{ax}\PY{o}{.}\PY{n}{set\PYZus{}xlabel}\PY{p}{(}\PY{l+s+s1}{\PYZsq{}}\PY{l+s+s1}{OrderQuantity}\PY{l+s+s1}{\PYZsq{}}\PY{p}{)}
\PY{n}{ax}\PY{o}{.}\PY{n}{set\PYZus{}ylabel}\PY{p}{(}\PY{l+s+s1}{\PYZsq{}}\PY{l+s+s1}{Demand}\PY{l+s+s1}{\PYZsq{}}\PY{p}{)}
\PY{n}{ax}\PY{o}{.}\PY{n}{set\PYZus{}zlabel}\PY{p}{(}\PY{l+s+s1}{\PYZsq{}}\PY{l+s+s1}{Profit}\PY{l+s+s1}{\PYZsq{}}\PY{p}{)}

\PY{n}{plt}\PY{o}{.}\PY{n}{show}\PY{p}{(}\PY{p}{)}\PY{p}{;}
\end{Verbatim}
\end{tcolorbox}

    \begin{center}
    \adjustimage{max size={0.9\linewidth}{0.9\paperheight}}{output_50_0.png}
    \end{center}
    { \hspace*{\fill} \\}
    
    \hypertarget{everything-is-an-object-in-python}{%
\subsection{Everything is an object in
Python}\label{everything-is-an-object-in-python}}

So far we have used standard procedural Python for creating and using
our model. However, we want to explore using an object-oriented approach
to doing the same things we just did. Before doing that, we need to
review some basics regarding object-oriented programming.

Objects are the ``things'' we create, use, change and share in our
Python programs. Objects contain both data about the object and
functions for using and changing the object. If you've programmed in
Excel VBA, you've already worked with objects and their various
properties (data) and methods (functions). For example, consider a
\texttt{Worksheet} object. One of its properties is \texttt{Name} and we
can reference it with \texttt{Worksheet.Name}. To copy a worksheet, we
use \texttt{Worksheet.Copy} - \texttt{Copy} is a method.

\hypertarget{list-objects}{%
\subsubsection{List objects}\label{list-objects}}

In Python, everything is an object and we've already been using objects
in this notebook. It's probably not surprising that something like an
array or a list are objects. Let's see what kind of objects they are.

    \begin{tcolorbox}[breakable, size=fbox, boxrule=1pt, pad at break*=1mm,colback=cellbackground, colframe=cellborder]
\prompt{In}{incolor}{ }{\boxspacing}
\begin{Verbatim}[commandchars=\\\{\}]
\PY{n+nb}{type}\PY{p}{(}\PY{n}{order\PYZus{}quantity\PYZus{}range}\PY{p}{)}
\end{Verbatim}
\end{tcolorbox}

    \begin{tcolorbox}[breakable, size=fbox, boxrule=1pt, pad at break*=1mm,colback=cellbackground, colframe=cellborder]
\prompt{In}{incolor}{ }{\boxspacing}
\begin{Verbatim}[commandchars=\\\{\}]
\PY{n+nb}{type}\PY{p}{(}\PY{n}{data\PYZus{}table\PYZus{}1}\PY{p}{)}
\end{Verbatim}
\end{tcolorbox}

    Let's explore some list methods. To start, we will create a list based
on the \texttt{order\_quantity\_range} and then do a few things with the
new list.

    \begin{tcolorbox}[breakable, size=fbox, boxrule=1pt, pad at break*=1mm,colback=cellbackground, colframe=cellborder]
\prompt{In}{incolor}{ }{\boxspacing}
\begin{Verbatim}[commandchars=\\\{\}]
\PY{n}{oq\PYZus{}list} \PY{o}{=} \PY{n+nb}{list}\PY{p}{(}\PY{n}{order\PYZus{}quantity\PYZus{}range}\PY{p}{)}
\PY{n+nb}{print}\PY{p}{(}\PY{n}{oq\PYZus{}list}\PY{p}{)}
\PY{n+nb}{print}\PY{p}{(}\PY{n+nb}{type}\PY{p}{(}\PY{n}{oq\PYZus{}list}\PY{p}{)}\PY{p}{)}
\end{Verbatim}
\end{tcolorbox}

    A Python \texttt{class} is the template from which we create individual
objects, called \emph{instances} in our programs. The \texttt{list}
class lets us create list objects. Python has many built in classes with
numerous predefined methods and properties.

    \begin{tcolorbox}[breakable, size=fbox, boxrule=1pt, pad at break*=1mm,colback=cellbackground, colframe=cellborder]
\prompt{In}{incolor}{ }{\boxspacing}
\begin{Verbatim}[commandchars=\\\{\}]
\PY{c+c1}{\PYZsh{} Append a new item to end of list}
\PY{n}{oq\PYZus{}list}\PY{o}{.}\PY{n}{append}\PY{p}{(}\PY{l+m+mi}{350}\PY{p}{)}
\PY{n+nb}{print}\PY{p}{(}\PY{n}{oq\PYZus{}list}\PY{p}{)}
\end{Verbatim}
\end{tcolorbox}

    \begin{tcolorbox}[breakable, size=fbox, boxrule=1pt, pad at break*=1mm,colback=cellbackground, colframe=cellborder]
\prompt{In}{incolor}{ }{\boxspacing}
\begin{Verbatim}[commandchars=\\\{\}]
\PY{c+c1}{\PYZsh{} Reverse the order of our list}
\PY{n}{oq\PYZus{}list}\PY{o}{.}\PY{n}{reverse}\PY{p}{(}\PY{p}{)}
\PY{n+nb}{print}\PY{p}{(}\PY{n}{oq\PYZus{}list}\PY{p}{)}
\end{Verbatim}
\end{tcolorbox}

    \begin{tcolorbox}[breakable, size=fbox, boxrule=1pt, pad at break*=1mm,colback=cellbackground, colframe=cellborder]
\prompt{In}{incolor}{ }{\boxspacing}
\begin{Verbatim}[commandchars=\\\{\}]
\PY{c+c1}{\PYZsh{} Reverse it back}
\PY{n}{oq\PYZus{}list}\PY{o}{.}\PY{n}{reverse}\PY{p}{(}\PY{p}{)}
\PY{n+nb}{print}\PY{p}{(}\PY{n}{oq\PYZus{}list}\PY{p}{)}
\end{Verbatim}
\end{tcolorbox}

    Notice that when we use the \texttt{append} and \texttt{reverse}
functions we need to include parentheses just as we do when calling any
function. In this case, we aren't passing any arguments into the
function, but often we will.

    \begin{tcolorbox}[breakable, size=fbox, boxrule=1pt, pad at break*=1mm,colback=cellbackground, colframe=cellborder]
\prompt{In}{incolor}{ }{\boxspacing}
\begin{Verbatim}[commandchars=\\\{\}]
\PY{c+c1}{\PYZsh{} Use sort() method to do a reverse sort of our list}
\PY{n}{oq\PYZus{}list}\PY{o}{.}\PY{n}{sort}\PY{p}{(}\PY{n}{reverse}\PY{o}{=}\PY{k+kc}{True}\PY{p}{)}
\PY{n+nb}{print}\PY{p}{(}\PY{n}{oq\PYZus{}list}\PY{p}{)}
\end{Verbatim}
\end{tcolorbox}

    What if we forget the parentheses?

    \begin{tcolorbox}[breakable, size=fbox, boxrule=1pt, pad at break*=1mm,colback=cellbackground, colframe=cellborder]
\prompt{In}{incolor}{ }{\boxspacing}
\begin{Verbatim}[commandchars=\\\{\}]
\PY{n+nb}{print}\PY{p}{(}\PY{n}{oq\PYZus{}list}\PY{o}{.}\PY{n}{reverse}\PY{p}{)}
\end{Verbatim}
\end{tcolorbox}

    Python just tells us that \texttt{reverse} is a function. It's also
telling us that \texttt{reverse}, a function, is also an object. Yes,
functions are objects, too. That means we can do things like pass them
to other functions or modify some of their properties. Why would we ever
want to do this? We'll get to that.

    I'm sure you already know how to get items from a list by their index
number.

    \begin{tcolorbox}[breakable, size=fbox, boxrule=1pt, pad at break*=1mm,colback=cellbackground, colframe=cellborder]
\prompt{In}{incolor}{ }{\boxspacing}
\begin{Verbatim}[commandchars=\\\{\}]
\PY{c+c1}{\PYZsh{} Get the first element from the list}
\PY{n+nb}{print}\PY{p}{(}\PY{n}{oq\PYZus{}list}\PY{p}{[}\PY{l+m+mi}{0}\PY{p}{]}\PY{p}{)}
\end{Verbatim}
\end{tcolorbox}

    The bracket notation is really just a convenient shorthand for calling
one of the built in methods of list objects. We can use Python
\texttt{dir} function to see all of the attributes of any object.

    \begin{tcolorbox}[breakable, size=fbox, boxrule=1pt, pad at break*=1mm,colback=cellbackground, colframe=cellborder]
\prompt{In}{incolor}{ }{\boxspacing}
\begin{Verbatim}[commandchars=\\\{\}]
\PY{n+nb}{print}\PY{p}{(}\PY{n+nb}{dir}\PY{p}{(}\PY{n}{oq\PYZus{}list}\PY{p}{)}\PY{p}{)}
\end{Verbatim}
\end{tcolorbox}

    Many of the attributes have both leading and trailing double
underscores, known as ``dunder'' attributes. These are just special
names reserved by Python for particular uses. In Python, you'll see use
of both single and double underscores, as both leading and trailing
characters. To dig deeper, check out
https://dbader.org/blog/meaning-of-underscores-in-python.

    In our example, using brackets to access a list item is really just a
different way of calling the \texttt{\_\_getitem\_\_} dunder method.

    \begin{tcolorbox}[breakable, size=fbox, boxrule=1pt, pad at break*=1mm,colback=cellbackground, colframe=cellborder]
\prompt{In}{incolor}{ }{\boxspacing}
\begin{Verbatim}[commandchars=\\\{\}]
\PY{n+nb}{print}\PY{p}{(}\PY{n}{oq\PYZus{}list}\PY{o}{.}\PY{n+nf+fm}{\PYZus{}\PYZus{}getitem\PYZus{}\PYZus{}}\PY{p}{(}\PY{l+m+mi}{0}\PY{p}{)}\PY{p}{)}
\end{Verbatim}
\end{tcolorbox}

    Also notice that all of these functions worked \emph{in place} - they
operated directly on the object \texttt{oq\_list} and even modified it
for the sort related methods. Some methods will create a new instance of
the object and if you want to ``save'' it, you'll have to assign the
result back to your original object or create a new variable. If you've
used R, you might recall that the dplyr package \textbf{never} modifies
dataframes in place. The dplyr functions (e.g.~\texttt{select},
\texttt{mutate}, \texttt{filter}, \ldots{}) always return new objects
and we have to ``catch the results'' by assigning to a variable if we
wanted to actually modify our dataframe.

Returning to our list example, we'll make a copy and explore a bit.

    \begin{tcolorbox}[breakable, size=fbox, boxrule=1pt, pad at break*=1mm,colback=cellbackground, colframe=cellborder]
\prompt{In}{incolor}{ }{\boxspacing}
\begin{Verbatim}[commandchars=\\\{\}]
\PY{n}{new\PYZus{}list} \PY{o}{=} \PY{n}{oq\PYZus{}list}\PY{o}{.}\PY{n}{copy}\PY{p}{(}\PY{p}{)}    \PY{c+c1}{\PYZsh{} Make a copy and save to new variable}
\PY{n+nb}{print}\PY{p}{(}\PY{n}{new\PYZus{}list}\PY{p}{)}              \PY{c+c1}{\PYZsh{} Check out the new variable}
\PY{n+nb}{print}\PY{p}{(}\PY{n}{new\PYZus{}list} \PY{o}{==} \PY{n}{oq\PYZus{}list}\PY{p}{)}   \PY{c+c1}{\PYZsh{} Does the new variable have the same value as the old variable?}
\PY{n+nb}{print}\PY{p}{(}\PY{n+nb}{type}\PY{p}{(}\PY{n}{new\PYZus{}list}\PY{p}{)} \PY{o}{==} \PY{n+nb}{type}\PY{p}{(}\PY{n}{oq\PYZus{}list}\PY{p}{)}\PY{p}{)}  \PY{c+c1}{\PYZsh{} Are the two variables of the same type?}
\PY{n+nb}{print}\PY{p}{(}\PY{n}{new\PYZus{}list} \PY{o+ow}{is} \PY{n}{oq\PYZus{}list}\PY{p}{)}   \PY{c+c1}{\PYZsh{} Is the new variable actually the same object as the old variable?}
\PY{n+nb}{print}\PY{p}{(}\PY{n+nb}{id}\PY{p}{(}\PY{n}{oq\PYZus{}list}\PY{p}{)}\PY{p}{,} \PY{n+nb}{id}\PY{p}{(}\PY{n}{new\PYZus{}list}\PY{p}{)}\PY{p}{)}  \PY{c+c1}{\PYZsh{} Every object has its own unique id number. Use id() to see it.}
\end{Verbatim}
\end{tcolorbox}

    The two lists have the same value and are of the same datatype, but are
\textbf{not} the same object.

    \textbf{QUESTION} If we reversed one of the two lists, would they still
be considered to have the same value? In other words, does order matter
with lists?

    Care must be taken when copying lists.

    \begin{tcolorbox}[breakable, size=fbox, boxrule=1pt, pad at break*=1mm,colback=cellbackground, colframe=cellborder]
\prompt{In}{incolor}{ }{\boxspacing}
\begin{Verbatim}[commandchars=\\\{\}]
\PY{n+nb}{print}\PY{p}{(}\PY{n}{oq\PYZus{}list}\PY{p}{)}
\end{Verbatim}
\end{tcolorbox}

    \begin{tcolorbox}[breakable, size=fbox, boxrule=1pt, pad at break*=1mm,colback=cellbackground, colframe=cellborder]
\prompt{In}{incolor}{ }{\boxspacing}
\begin{Verbatim}[commandchars=\\\{\}]
\PY{c+c1}{\PYZsh{} Instead of using copy method, just setting new variable value equal to existing list}
\PY{n}{new\PYZus{}oq\PYZus{}list} \PY{o}{=} \PY{n}{oq\PYZus{}list} 
\PY{n+nb}{print}\PY{p}{(}\PY{n}{new\PYZus{}oq\PYZus{}list}\PY{p}{)}
\PY{n}{oq\PYZus{}list}\PY{o}{.}\PY{n}{reverse}\PY{p}{(}\PY{p}{)} \PY{c+c1}{\PYZsh{} I\PYZsq{}m reversing the ORIGINAL list}
\PY{n+nb}{print}\PY{p}{(}\PY{n}{new\PYZus{}oq\PYZus{}list}\PY{p}{)}
\PY{n+nb}{print}\PY{p}{(}\PY{n+nb}{id}\PY{p}{(}\PY{n}{oq\PYZus{}list}\PY{p}{)}\PY{p}{,} \PY{n+nb}{id}\PY{p}{(}\PY{n}{new\PYZus{}oq\PYZus{}list}\PY{p}{)}\PY{p}{)}
\end{Verbatim}
\end{tcolorbox}

    Yikes! As you can see from the \texttt{id} numbers, both list variables
point to the same object in memory. Changing one list, changes the other
- they're the same object. So, how do you safely copy lists? You use the
\texttt{copy} method like we did above.

    Let's look at the attributes of a \texttt{numpy.array} object and see
how that compares to a list. They are both data structures that allow us
to store a bunch of values and get at them via an index number. What
differences do you think these two data structures might have?

    \begin{tcolorbox}[breakable, size=fbox, boxrule=1pt, pad at break*=1mm,colback=cellbackground, colframe=cellborder]
\prompt{In}{incolor}{ }{\boxspacing}
\begin{Verbatim}[commandchars=\\\{\}]
\PY{n+nb}{dir}\PY{p}{(}\PY{n}{order\PYZus{}quantity\PYZus{}range}\PY{p}{)}
\end{Verbatim}
\end{tcolorbox}

    Wow,
\href{https://jakevdp.github.io/PythonDataScienceHandbook/02.02-the-basics-of-numpy-arrays.html}{numpy
arrays} have a whole lot more going on under the hood than do lists.
They play a huge role in doing data science work in Python with many
packages using numpy to store and work with data. Notice that there are
various mathematical methods such as \texttt{sum} and \texttt{mean} and
methods to convert arrays to lists and strings using the obviously named
methods. We see that there is a \texttt{dtype} property - it tells us
the data type of the array. Our array stores 64bit integers.

    \begin{tcolorbox}[breakable, size=fbox, boxrule=1pt, pad at break*=1mm,colback=cellbackground, colframe=cellborder]
\prompt{In}{incolor}{ }{\boxspacing}
\begin{Verbatim}[commandchars=\\\{\}]
\PY{n+nb}{print}\PY{p}{(}\PY{n}{order\PYZus{}quantity\PYZus{}range}\PY{o}{.}\PY{n}{dtype}\PY{p}{)}
\end{Verbatim}
\end{tcolorbox}

    \textbf{QUESTIONS} Why don't we use parens with \texttt{dtype}? Why
don't list objects have a \texttt{dtype} property?

    \hypertarget{what-about-numbers}{%
\subsubsection{What about numbers?}\label{what-about-numbers}}

Are things like integers and floating point numbers actually objects?

    \begin{tcolorbox}[breakable, size=fbox, boxrule=1pt, pad at break*=1mm,colback=cellbackground, colframe=cellborder]
\prompt{In}{incolor}{ }{\boxspacing}
\begin{Verbatim}[commandchars=\\\{\}]
\PY{n+nb}{print}\PY{p}{(}\PY{n}{unit\PYZus{}cost}\PY{p}{)}
\PY{n+nb}{print}\PY{p}{(}\PY{n+nb}{type}\PY{p}{(}\PY{n}{unit\PYZus{}cost}\PY{p}{)}\PY{p}{)}
\PY{n+nb}{print}\PY{p}{(}\PY{n+nb}{dir}\PY{p}{(}\PY{n}{unit\PYZus{}cost}\PY{p}{)}\PY{p}{)}
\end{Verbatim}
\end{tcolorbox}

    \begin{tcolorbox}[breakable, size=fbox, boxrule=1pt, pad at break*=1mm,colback=cellbackground, colframe=cellborder]
\prompt{In}{incolor}{ }{\boxspacing}
\begin{Verbatim}[commandchars=\\\{\}]
\PY{n}{unit\PYZus{}cost}\PY{o}{.}\PY{n}{is\PYZus{}integer}\PY{p}{(}\PY{p}{)}
\end{Verbatim}
\end{tcolorbox}

    \begin{tcolorbox}[breakable, size=fbox, boxrule=1pt, pad at break*=1mm,colback=cellbackground, colframe=cellborder]
\prompt{In}{incolor}{ }{\boxspacing}
\begin{Verbatim}[commandchars=\\\{\}]
\PY{n}{unit\PYZus{}cost}\PY{o}{.}\PY{n}{hex}\PY{p}{(}\PY{p}{)}
\end{Verbatim}
\end{tcolorbox}

    Yep. Try out some of the attributes.

    \hypertarget{writing-our-own-classes}{%
\subsubsection{Writing our own classes}\label{writing-our-own-classes}}

While Python has many useful built in classes, true
\href{https://en.wikipedia.org/wiki/Object-oriented_programming}{object-oriented
programming} (OOP) involves creating our own classes from which we can
create object instances that are useful for the problem at hand. Python
is an OOP language but it does \textbf{not} force you to create classes
if you don't want to. You can write any blend you'd like of
\href{https://en.wikipedia.org/wiki/Procedural_programming}{procedural},
object oriented and
\href{https://en.wikipedia.org/wiki/Functional_programming}{functional
programming}. We aren't going to get into debates about which
programming paradigm is the ``best''. There are times when OOP makes a
lot of sense, and for those times, we'll use it. Other times, simply
creating useful functions (such as the \texttt{bookstore\_profit}
function we used earlier) is the appropriate thing to do.

Many business analysts who learn Excel VBA, don't realize that
\textbf{you can actually create classes in VBA}. Most stick with using
the built in Excel objects.

    \hypertarget{an-oo-version-of-the-bookstore-model}{%
\subsection{An OO version of the Bookstore
Model}\label{an-oo-version-of-the-bookstore-model}}

Let's create an OO version of the Bookstore Model and figure out how to
do n-way data tables just as we did with the non-OO version of the
model. Then we'll go on and figure out how to duplicate the
functionality of Excel's Goal Seek tool in Python. It's not clear
whether the non-OO or the OO version of our model will ultimately prove
best for this use case. Regardless, we'll learn a few things along the
way.

As an aside, there's a very useful site aimed at using Python for
quantitative economics and finance - https://quantecon.org/. Within
there are some terrific notebooks on various Python topics. Here are two
related to OOP:

\begin{itemize}
\tightlist
\item
  https://python-programming.quantecon.org/oop\_intro.html
\item
  https://python-programming.quantecon.org/python\_oop.html
\end{itemize}

    \hypertarget{initial-design-of-the-bookstoremodel-class}{%
\subsubsection{Initial design of the BookstoreModel
class}\label{initial-design-of-the-bookstoremodel-class}}

It seems like we would want all of the base inputs such as unit cost,
selling price, unit refund, order quantity and demand to be attributes
(properties) of the class. Here's the start of our class definition
code:

    \begin{tcolorbox}[breakable, size=fbox, boxrule=1pt, pad at break*=1mm,colback=cellbackground, colframe=cellborder]
\prompt{In}{incolor}{24}{\boxspacing}
\begin{Verbatim}[commandchars=\\\{\}]
\PY{k}{class} \PY{n+nc}{BookstoreModel}\PY{p}{(}\PY{p}{)}\PY{p}{:}
    \PY{k}{def} \PY{n+nf+fm}{\PYZus{}\PYZus{}init\PYZus{}\PYZus{}}\PY{p}{(}\PY{n+nb+bp}{self}\PY{p}{,} \PY{n}{unit\PYZus{}cost}\PY{p}{,} \PY{n}{selling\PYZus{}price}\PY{p}{,} \PY{n}{unit\PYZus{}refund}\PY{p}{,} \PY{n}{order\PYZus{}quantity}\PY{p}{,} \PY{n}{demand}\PY{p}{)}\PY{p}{:}
        \PY{n+nb+bp}{self}\PY{o}{.}\PY{n}{unit\PYZus{}cost} \PY{o}{=} \PY{n}{unit\PYZus{}cost}
        \PY{n+nb+bp}{self}\PY{o}{.}\PY{n}{selling\PYZus{}price} \PY{o}{=} \PY{n}{selling\PYZus{}price}
        \PY{n+nb+bp}{self}\PY{o}{.}\PY{n}{unit\PYZus{}refund} \PY{o}{=} \PY{n}{unit\PYZus{}refund}
        \PY{n+nb+bp}{self}\PY{o}{.}\PY{n}{order\PYZus{}quantity} \PY{o}{=} \PY{n}{order\PYZus{}quantity}
        \PY{n+nb+bp}{self}\PY{o}{.}\PY{n}{demand} \PY{o}{=} \PY{n}{demand}
\end{Verbatim}
\end{tcolorbox}

    Some important things of note:

\begin{itemize}
\tightlist
\item
  we use the keyword \texttt{class} followed by the name of our class.
\item
  by convention, class names are capitalized (each word if multi-word).
\item
  the class declaration looks a little like defining a function but we
  use \texttt{class} instead of \texttt{def}.
\item
  there's nothing in the parentheses following the class name. We'll
  defer talking about this for now.
\item
  there's a single method (function) defined within the class and it is
  called \texttt{\_\_init\_\_}. From earlier in this document we know
  that this must be a built in dunder method that has some special
  purpose. From its name you've probably already guessed that it's the
  function called when a new object instance is first created. If you've
  programmed in Java or C++, this is kind of like a constructor.
\item
  the first argument to the \texttt{\_\_init\_\_} function is
  \texttt{self} and represents the object instance being created.
\item
  the remaining arguments to the \texttt{\_\_init\_\_} function are
  values that we want to use in the function. In this case, we simply
  want to create object properties corresponding to the base inputs for
  the bookstore model.
\item
  the rest of the lines within the \texttt{\_\_init\_\_} function are
  initializing our object properties to the values passed in.
\item
  in order to reference (i.e.~to get the value or set the value) an
  object property, we always preface the property with the \texttt{self}
  object, e.g., \texttt{self.unit\_cost}.
\item
  if you've programmed in Java you might be used to using ``setters and
  getters''. In general
  \href{https://www.python-course.eu/python3_properties.php}{we don't do
  that in Python} and instead just reference the property directly using
  dot notation.
\end{itemize}

    Well, our class isn't super useful yet, but let's try it out. Make sure
you run the code chunk above that defines the BookstoreModel class.

    \begin{tcolorbox}[breakable, size=fbox, boxrule=1pt, pad at break*=1mm,colback=cellbackground, colframe=cellborder]
\prompt{In}{incolor}{25}{\boxspacing}
\begin{Verbatim}[commandchars=\\\{\}]
\PY{c+c1}{\PYZsh{} Reset the base inputs}
\PY{n}{unit\PYZus{}cost} \PY{o}{=} \PY{l+m+mf}{7.50}
\PY{n}{selling\PYZus{}price} \PY{o}{=} \PY{l+m+mf}{10.00}
\PY{n}{unit\PYZus{}refund} \PY{o}{=} \PY{l+m+mf}{2.50}
\PY{n}{demand} \PY{o}{=} \PY{l+m+mi}{193}
\PY{n}{order\PYZus{}quantity} \PY{o}{=} \PY{l+m+mi}{200}

\PY{c+c1}{\PYZsh{} Create a new BookstoreModel object}
\PY{n}{model\PYZus{}1} \PY{o}{=} \PY{n}{BookstoreModel}\PY{p}{(}\PY{n}{unit\PYZus{}cost}\PY{p}{,} \PY{n}{selling\PYZus{}price}\PY{p}{,} \PY{n}{unit\PYZus{}refund}\PY{p}{,} \PY{n}{order\PYZus{}quantity}\PY{p}{,} \PY{n}{demand}\PY{p}{)}
\end{Verbatim}
\end{tcolorbox}

    Let's print out one of the properties.

    \begin{tcolorbox}[breakable, size=fbox, boxrule=1pt, pad at break*=1mm,colback=cellbackground, colframe=cellborder]
\prompt{In}{incolor}{26}{\boxspacing}
\begin{Verbatim}[commandchars=\\\{\}]
\PY{n+nb}{print}\PY{p}{(}\PY{n}{model\PYZus{}1}\PY{o}{.}\PY{n}{unit\PYZus{}cost}\PY{p}{)}
\end{Verbatim}
\end{tcolorbox}

    \begin{Verbatim}[commandchars=\\\{\}]
7.5
    \end{Verbatim}

    What happens if we try to print the model object itself?

    \begin{tcolorbox}[breakable, size=fbox, boxrule=1pt, pad at break*=1mm,colback=cellbackground, colframe=cellborder]
\prompt{In}{incolor}{27}{\boxspacing}
\begin{Verbatim}[commandchars=\\\{\}]
\PY{n+nb}{print}\PY{p}{(}\PY{n}{model\PYZus{}1}\PY{p}{)}
\end{Verbatim}
\end{tcolorbox}

    \begin{Verbatim}[commandchars=\\\{\}]
<\_\_main\_\_.BookstoreModel object at 0x7fed59a46290>
    \end{Verbatim}

    Ok, not the most useful but we do see that we've got an object of the
right type and that the object lives in the \texttt{\_\_main\_\_}
\emph{namespace}. Namespaces are important in Python. They're even
mentioned in the Zen of Python. You can
\href{https://realpython.com/python-namespaces-scope/}{learn more about
namespaces here}.

    \begin{tcolorbox}[breakable, size=fbox, boxrule=1pt, pad at break*=1mm,colback=cellbackground, colframe=cellborder]
\prompt{In}{incolor}{28}{\boxspacing}
\begin{Verbatim}[commandchars=\\\{\}]
\PY{k+kn}{import} \PY{n+nn}{this}
\end{Verbatim}
\end{tcolorbox}

    \begin{Verbatim}[commandchars=\\\{\}]
The Zen of Python, by Tim Peters

Beautiful is better than ugly.
Explicit is better than implicit.
Simple is better than complex.
Complex is better than complicated.
Flat is better than nested.
Sparse is better than dense.
Readability counts.
Special cases aren't special enough to break the rules.
Although practicality beats purity.
Errors should never pass silently.
Unless explicitly silenced.
In the face of ambiguity, refuse the temptation to guess.
There should be one-- and preferably only one --obvious way to do it.
Although that way may not be obvious at first unless you're Dutch.
Now is better than never.
Although never is often better than *right* now.
If the implementation is hard to explain, it's a bad idea.
If the implementation is easy to explain, it may be a good idea.
Namespaces are one honking great idea -- let's do more of those!
    \end{Verbatim}

    If you look back at our \texttt{bookstore\_profit} function, you'll see
that it first computes the various cost and revenue components and then
computes profit from that. So, it would seem useful to add some methods
to our BookstoreModel class that computed these things.

    \begin{tcolorbox}[breakable, size=fbox, boxrule=1pt, pad at break*=1mm,colback=cellbackground, colframe=cellborder]
\prompt{In}{incolor}{29}{\boxspacing}
\begin{Verbatim}[commandchars=\\\{\}]
\PY{k}{class} \PY{n+nc}{BookstoreModel}\PY{p}{(}\PY{p}{)}\PY{p}{:}
    \PY{k}{def} \PY{n+nf+fm}{\PYZus{}\PYZus{}init\PYZus{}\PYZus{}}\PY{p}{(}\PY{n+nb+bp}{self}\PY{p}{,} \PY{n}{unit\PYZus{}cost}\PY{p}{,} \PY{n}{selling\PYZus{}price}\PY{p}{,} \PY{n}{unit\PYZus{}refund}\PY{p}{,} \PY{n}{order\PYZus{}quantity}\PY{p}{,} \PY{n}{demand}\PY{p}{)}\PY{p}{:}
        \PY{n+nb+bp}{self}\PY{o}{.}\PY{n}{unit\PYZus{}cost} \PY{o}{=} \PY{n}{unit\PYZus{}cost}
        \PY{n+nb+bp}{self}\PY{o}{.}\PY{n}{selling\PYZus{}price} \PY{o}{=} \PY{n}{selling\PYZus{}price}
        \PY{n+nb+bp}{self}\PY{o}{.}\PY{n}{unit\PYZus{}refund} \PY{o}{=} \PY{n}{unit\PYZus{}refund}
        \PY{n+nb+bp}{self}\PY{o}{.}\PY{n}{order\PYZus{}quantity} \PY{o}{=} \PY{n}{order\PYZus{}quantity}
        \PY{n+nb+bp}{self}\PY{o}{.}\PY{n}{demand} \PY{o}{=} \PY{n}{demand}
        
    \PY{k}{def} \PY{n+nf}{order\PYZus{}cost}\PY{p}{(}\PY{n+nb+bp}{self}\PY{p}{)}\PY{p}{:}
        \PY{l+s+sd}{\PYZdq{}\PYZdq{}\PYZdq{}Compute total order cost\PYZdq{}\PYZdq{}\PYZdq{}}
        \PY{k}{return} \PY{n+nb+bp}{self}\PY{o}{.}\PY{n}{unit\PYZus{}cost} \PY{o}{*} \PY{n+nb+bp}{self}\PY{o}{.}\PY{n}{order\PYZus{}quantity}
    
    \PY{k}{def} \PY{n+nf}{sales\PYZus{}revenue}\PY{p}{(}\PY{n+nb+bp}{self}\PY{p}{)}\PY{p}{:}
        \PY{l+s+sd}{\PYZdq{}\PYZdq{}\PYZdq{}Compute sales revenue\PYZdq{}\PYZdq{}\PYZdq{}}
        \PY{k}{return} \PY{n}{np}\PY{o}{.}\PY{n}{minimum}\PY{p}{(}\PY{n+nb+bp}{self}\PY{o}{.}\PY{n}{order\PYZus{}quantity}\PY{p}{,} \PY{n+nb+bp}{self}\PY{o}{.}\PY{n}{demand}\PY{p}{)} \PY{o}{*} \PY{n+nb+bp}{self}\PY{o}{.}\PY{n}{selling\PYZus{}price}
    
    \PY{k}{def} \PY{n+nf}{refund\PYZus{}revenue}\PY{p}{(}\PY{n+nb+bp}{self}\PY{p}{)}\PY{p}{:}
        \PY{l+s+sd}{\PYZdq{}\PYZdq{}\PYZdq{}Compute revenue from refunds for unsold items\PYZdq{}\PYZdq{}\PYZdq{}}
        \PY{k}{return} \PY{n}{np}\PY{o}{.}\PY{n}{maximum}\PY{p}{(}\PY{l+m+mi}{0}\PY{p}{,} \PY{n+nb+bp}{self}\PY{o}{.}\PY{n}{order\PYZus{}quantity} \PY{o}{\PYZhy{}} \PY{n+nb+bp}{self}\PY{o}{.}\PY{n}{demand}\PY{p}{)} \PY{o}{*} \PY{n+nb+bp}{self}\PY{o}{.}\PY{n}{unit\PYZus{}refund}
    
    \PY{k}{def} \PY{n+nf}{profit}\PY{p}{(}\PY{n+nb+bp}{self}\PY{p}{)}\PY{p}{:}
        \PY{l+s+sd}{\PYZsq{}\PYZsq{}\PYZsq{}}
\PY{l+s+sd}{        Compute profit}
\PY{l+s+sd}{        \PYZsq{}\PYZsq{}\PYZsq{}}
        \PY{n}{profit} \PY{o}{=} \PY{n+nb+bp}{self}\PY{o}{.}\PY{n}{sales\PYZus{}revenue}\PY{p}{(}\PY{p}{)} \PY{o}{+} \PY{n+nb+bp}{self}\PY{o}{.}\PY{n}{refund\PYZus{}revenue}\PY{p}{(}\PY{p}{)} \PY{o}{\PYZhy{}} \PY{n+nb+bp}{self}\PY{o}{.}\PY{n}{order\PYZus{}cost}\PY{p}{(}\PY{p}{)}
        \PY{k}{return} \PY{n}{profit}
\end{Verbatim}
\end{tcolorbox}

    A few important things to note: * we added four new object functions
(methods) * each method only takes the \texttt{self} object as an input
argument. We don't pass in the properties such as \texttt{unit\_cost} or
\texttt{order\_quantity}. We don't need to since \texttt{self} knows
about these things via its properties. They are \emph{encapsulated}
within the object. * within each method we have to use
\texttt{self.someproperty} in our expressions * the \texttt{profit}
method calls some of the other methods and when it does, it prefaces the
method with \texttt{self.} and follows it with empty parens. The
\texttt{self} object is implicitly passed. * each method returns a value
* each method has a simple docstring

Before we can test out our enhanced class, we need to run the code cell
above and we need to create a new object instance based on the updated
class definition.

    \begin{tcolorbox}[breakable, size=fbox, boxrule=1pt, pad at break*=1mm,colback=cellbackground, colframe=cellborder]
\prompt{In}{incolor}{30}{\boxspacing}
\begin{Verbatim}[commandchars=\\\{\}]
\PY{c+c1}{\PYZsh{} Create a new BookstoreModel object}
\PY{n}{model\PYZus{}2} \PY{o}{=} \PY{n}{BookstoreModel}\PY{p}{(}\PY{n}{unit\PYZus{}cost}\PY{p}{,} \PY{n}{selling\PYZus{}price}\PY{p}{,} \PY{n}{unit\PYZus{}refund}\PY{p}{,} \PY{n}{order\PYZus{}quantity}\PY{p}{,} \PY{n}{demand}\PY{p}{)}
\end{Verbatim}
\end{tcolorbox}

    \begin{tcolorbox}[breakable, size=fbox, boxrule=1pt, pad at break*=1mm,colback=cellbackground, colframe=cellborder]
\prompt{In}{incolor}{31}{\boxspacing}
\begin{Verbatim}[commandchars=\\\{\}]
\PY{n+nb}{print}\PY{p}{(}\PY{n}{model\PYZus{}2}\PY{o}{.}\PY{n}{sales\PYZus{}revenue}\PY{p}{(}\PY{p}{)}\PY{p}{)}
\PY{n+nb}{print}\PY{p}{(}\PY{n}{model\PYZus{}2}\PY{o}{.}\PY{n}{profit}\PY{p}{(}\PY{p}{)}\PY{p}{)}
\end{Verbatim}
\end{tcolorbox}

    \begin{Verbatim}[commandchars=\\\{\}]
1930.0
447.5
    \end{Verbatim}

    It would nice to have a way to print out a succinct summary of a
BookstoreModel object. Let's start by just listing out the properties.
We can use the Python \texttt{vars} function to help. This StackOverflow
post has some good info on getting a dictionary of object properties:

https://stackoverflow.com/questions/61517/python-dictionary-from-an-objects-fields

    \begin{tcolorbox}[breakable, size=fbox, boxrule=1pt, pad at break*=1mm,colback=cellbackground, colframe=cellborder]
\prompt{In}{incolor}{32}{\boxspacing}
\begin{Verbatim}[commandchars=\\\{\}]
\PY{n+nb}{print}\PY{p}{(}\PY{n+nb}{vars}\PY{p}{(}\PY{n}{model\PYZus{}2}\PY{p}{)}\PY{p}{)}
\end{Verbatim}
\end{tcolorbox}

    \begin{Verbatim}[commandchars=\\\{\}]
\{'unit\_cost': 7.5, 'selling\_price': 10.0, 'unit\_refund': 2.5, 'order\_quantity':
200, 'demand': 193\}
    \end{Verbatim}

    The Pythonic way for creating a string representation of an object is to
implement a \texttt{\_\_str\_\_} dunder function for our object. This
function needs to return a string. Notice that the \texttt{vars}
function returns a dictionary. You can learn more about
\texttt{\_\_str\_\_} and the related \texttt{\_\_repr\_\_} dunder
functions at https://dbader.org/blog/python-repr-vs-str. In a nutshell,
we use \texttt{\_\_str\_\_} to create ``pretty'' string representations
of an object for our user and \texttt{\_\_repr\_\_} for an unambiguous
string representation that includes the object type.

    \begin{tcolorbox}[breakable, size=fbox, boxrule=1pt, pad at break*=1mm,colback=cellbackground, colframe=cellborder]
\prompt{In}{incolor}{33}{\boxspacing}
\begin{Verbatim}[commandchars=\\\{\}]
\PY{k}{class} \PY{n+nc}{BookstoreModel}\PY{p}{(}\PY{p}{)}\PY{p}{:}
    \PY{k}{def} \PY{n+nf+fm}{\PYZus{}\PYZus{}init\PYZus{}\PYZus{}}\PY{p}{(}\PY{n+nb+bp}{self}\PY{p}{,} \PY{n}{unit\PYZus{}cost}\PY{p}{,} \PY{n}{selling\PYZus{}price}\PY{p}{,} \PY{n}{unit\PYZus{}refund}\PY{p}{,} \PY{n}{order\PYZus{}quantity}\PY{p}{,} \PY{n}{demand}\PY{p}{)}\PY{p}{:}
        \PY{n+nb+bp}{self}\PY{o}{.}\PY{n}{unit\PYZus{}cost} \PY{o}{=} \PY{n}{unit\PYZus{}cost}
        \PY{n+nb+bp}{self}\PY{o}{.}\PY{n}{selling\PYZus{}price} \PY{o}{=} \PY{n}{selling\PYZus{}price}
        \PY{n+nb+bp}{self}\PY{o}{.}\PY{n}{unit\PYZus{}refund} \PY{o}{=} \PY{n}{unit\PYZus{}refund}
        \PY{n+nb+bp}{self}\PY{o}{.}\PY{n}{order\PYZus{}quantity} \PY{o}{=} \PY{n}{order\PYZus{}quantity}
        \PY{n+nb+bp}{self}\PY{o}{.}\PY{n}{demand} \PY{o}{=} \PY{n}{demand}
        
    \PY{k}{def} \PY{n+nf}{order\PYZus{}cost}\PY{p}{(}\PY{n+nb+bp}{self}\PY{p}{)}\PY{p}{:}
        \PY{l+s+sd}{\PYZdq{}\PYZdq{}\PYZdq{}Compute total order cost\PYZdq{}\PYZdq{}\PYZdq{}}
        \PY{k}{return} \PY{n+nb+bp}{self}\PY{o}{.}\PY{n}{unit\PYZus{}cost} \PY{o}{*} \PY{n+nb+bp}{self}\PY{o}{.}\PY{n}{order\PYZus{}quantity}
    
    \PY{k}{def} \PY{n+nf}{sales\PYZus{}revenue}\PY{p}{(}\PY{n+nb+bp}{self}\PY{p}{)}\PY{p}{:}
        \PY{l+s+sd}{\PYZdq{}\PYZdq{}\PYZdq{}Compute sales revenue\PYZdq{}\PYZdq{}\PYZdq{}}
        \PY{k}{return} \PY{n}{np}\PY{o}{.}\PY{n}{minimum}\PY{p}{(}\PY{n+nb+bp}{self}\PY{o}{.}\PY{n}{order\PYZus{}quantity}\PY{p}{,} \PY{n+nb+bp}{self}\PY{o}{.}\PY{n}{demand}\PY{p}{)} \PY{o}{*} \PY{n+nb+bp}{self}\PY{o}{.}\PY{n}{selling\PYZus{}price}
    
    \PY{k}{def} \PY{n+nf}{refund\PYZus{}revenue}\PY{p}{(}\PY{n+nb+bp}{self}\PY{p}{)}\PY{p}{:}
        \PY{l+s+sd}{\PYZdq{}\PYZdq{}\PYZdq{}Compute revenue from refunds for unsold items\PYZdq{}\PYZdq{}\PYZdq{}}
        \PY{k}{return} \PY{n}{np}\PY{o}{.}\PY{n}{maximum}\PY{p}{(}\PY{l+m+mi}{0}\PY{p}{,} \PY{n+nb+bp}{self}\PY{o}{.}\PY{n}{order\PYZus{}quantity} \PY{o}{\PYZhy{}} \PY{n+nb+bp}{self}\PY{o}{.}\PY{n}{demand}\PY{p}{)} \PY{o}{*} \PY{n+nb+bp}{self}\PY{o}{.}\PY{n}{unit\PYZus{}refund}
    
    \PY{k}{def} \PY{n+nf}{profit}\PY{p}{(}\PY{n+nb+bp}{self}\PY{p}{)}\PY{p}{:}
        \PY{l+s+sd}{\PYZsq{}\PYZsq{}\PYZsq{}}
\PY{l+s+sd}{        Compute profit}
\PY{l+s+sd}{        \PYZsq{}\PYZsq{}\PYZsq{}}
        \PY{n}{profit} \PY{o}{=} \PY{n+nb+bp}{self}\PY{o}{.}\PY{n}{sales\PYZus{}revenue}\PY{p}{(}\PY{p}{)} \PY{o}{+} \PY{n+nb+bp}{self}\PY{o}{.}\PY{n}{refund\PYZus{}revenue}\PY{p}{(}\PY{p}{)} \PY{o}{\PYZhy{}} \PY{n+nb+bp}{self}\PY{o}{.}\PY{n}{order\PYZus{}cost}\PY{p}{(}\PY{p}{)}
        \PY{k}{return} \PY{n}{profit}
    
    \PY{k}{def} \PY{n+nf+fm}{\PYZus{}\PYZus{}str\PYZus{}\PYZus{}}\PY{p}{(}\PY{n+nb+bp}{self}\PY{p}{)}\PY{p}{:}
        \PY{l+s+sd}{\PYZdq{}\PYZdq{}\PYZdq{}}
\PY{l+s+sd}{        String representation of bookstore inputs}
\PY{l+s+sd}{        \PYZdq{}\PYZdq{}\PYZdq{}}
        \PY{k}{return} \PY{n+nb}{str}\PY{p}{(}\PY{n+nb}{vars}\PY{p}{(}\PY{n+nb+bp}{self}\PY{p}{)}\PY{p}{)}
\end{Verbatim}
\end{tcolorbox}

    Notice we wrapped \texttt{vars} with \texttt{str} to convert the
dictionary to a string.

    \begin{tcolorbox}[breakable, size=fbox, boxrule=1pt, pad at break*=1mm,colback=cellbackground, colframe=cellborder]
\prompt{In}{incolor}{34}{\boxspacing}
\begin{Verbatim}[commandchars=\\\{\}]
\PY{c+c1}{\PYZsh{} Reset the base inputs}
\PY{n}{unit\PYZus{}cost} \PY{o}{=} \PY{l+m+mf}{7.50}
\PY{n}{selling\PYZus{}price} \PY{o}{=} \PY{l+m+mf}{10.00}
\PY{n}{unit\PYZus{}refund} \PY{o}{=} \PY{l+m+mf}{2.50}
\PY{n}{demand} \PY{o}{=} \PY{l+m+mi}{193}
\PY{n}{order\PYZus{}quantity} \PY{o}{=} \PY{l+m+mi}{200}

\PY{c+c1}{\PYZsh{} Create a new BookstoreModel object}
\PY{n}{model\PYZus{}3} \PY{o}{=} \PY{n}{BookstoreModel}\PY{p}{(}\PY{n}{unit\PYZus{}cost}\PY{p}{,} \PY{n}{selling\PYZus{}price}\PY{p}{,} \PY{n}{unit\PYZus{}refund}\PY{p}{,} \PY{n}{order\PYZus{}quantity}\PY{p}{,} \PY{n}{demand}\PY{p}{)}
\end{Verbatim}
\end{tcolorbox}

    \begin{tcolorbox}[breakable, size=fbox, boxrule=1pt, pad at break*=1mm,colback=cellbackground, colframe=cellborder]
\prompt{In}{incolor}{35}{\boxspacing}
\begin{Verbatim}[commandchars=\\\{\}]
\PY{n+nb}{print}\PY{p}{(}\PY{n}{model\PYZus{}3}\PY{p}{)}
\end{Verbatim}
\end{tcolorbox}

    \begin{Verbatim}[commandchars=\\\{\}]
\{'unit\_cost': 7.5, 'selling\_price': 10.0, 'unit\_refund': 2.5, 'order\_quantity':
200, 'demand': 193\}
    \end{Verbatim}

    \textbf{CHALLENGE 1} Modify the \texttt{\_\_str\_\_} function so that it
also shows the computed value of \texttt{order\_cost},
\texttt{sales\_revenue}, \texttt{refund\_revenue} and \texttt{profit}.

\textbf{CHALLENGE 2} Add a \texttt{lost\_sales} method to the
\texttt{BookstoreModel} class that returns the number of units that we
could not sell because we didn't order enough.

    \hypertarget{way-data-table-with-oo-model}{%
\subsubsection{1-way Data Table with OO
model}\label{way-data-table-with-oo-model}}

It seems entirely plausible that we could simply set one of the inputs
to an array of values and then call the \texttt{profit} method.

    \begin{tcolorbox}[breakable, size=fbox, boxrule=1pt, pad at break*=1mm,colback=cellbackground, colframe=cellborder]
\prompt{In}{incolor}{36}{\boxspacing}
\begin{Verbatim}[commandchars=\\\{\}]
\PY{c+c1}{\PYZsh{} Set property equal to an array}
\PY{n}{model\PYZus{}3}\PY{o}{.}\PY{n}{order\PYZus{}quantity} \PY{o}{=} \PY{n}{np}\PY{o}{.}\PY{n}{arange}\PY{p}{(}\PY{l+m+mi}{50}\PY{p}{,} \PY{l+m+mi}{301}\PY{p}{,} \PY{l+m+mi}{25}\PY{p}{)}
\PY{n}{model\PYZus{}3}\PY{o}{.}\PY{n}{order\PYZus{}quantity}
\end{Verbatim}
\end{tcolorbox}

            \begin{tcolorbox}[breakable, size=fbox, boxrule=.5pt, pad at break*=1mm, opacityfill=0]
\prompt{Out}{outcolor}{36}{\boxspacing}
\begin{Verbatim}[commandchars=\\\{\}]
array([ 50,  75, 100, 125, 150, 175, 200, 225, 250, 275, 300])
\end{Verbatim}
\end{tcolorbox}
        
    \begin{tcolorbox}[breakable, size=fbox, boxrule=1pt, pad at break*=1mm,colback=cellbackground, colframe=cellborder]
\prompt{In}{incolor}{37}{\boxspacing}
\begin{Verbatim}[commandchars=\\\{\}]
\PY{c+c1}{\PYZsh{} Can we compute an array of profits? Try it.}
\end{Verbatim}
\end{tcolorbox}

    What about a 2-way data table using the OO model? We can't use the same
direct array passing approach because of \ldots{}

    \begin{tcolorbox}[breakable, size=fbox, boxrule=1pt, pad at break*=1mm,colback=cellbackground, colframe=cellborder]
\prompt{In}{incolor}{38}{\boxspacing}
\begin{Verbatim}[commandchars=\\\{\}]
\PY{c+c1}{\PYZsh{} This won\PYZsq{}t work unless the two arrays are the same size \PYZhy{} can\PYZsq{}t broadcast.}
\PY{c+c1}{\PYZsh{} Also, even if same size, doesn\PYZsq{}t do all combinations, just aligned elements.}
\PY{n}{demand\PYZus{}range} \PY{o}{=} \PY{n}{np}\PY{o}{.}\PY{n}{arange}\PY{p}{(}\PY{l+m+mi}{70}\PY{p}{,} \PY{l+m+mi}{321}\PY{p}{,} \PY{l+m+mi}{5}\PY{p}{)}
\PY{n}{order\PYZus{}quantity\PYZus{}range} \PY{o}{=} \PY{n}{np}\PY{o}{.}\PY{n}{arange}\PY{p}{(}\PY{l+m+mi}{70}\PY{p}{,} \PY{l+m+mi}{321}\PY{p}{,} \PY{l+m+mi}{10}\PY{p}{)}    \PY{c+c1}{\PYZsh{} Broadcast error}
\PY{c+c1}{\PYZsh{} order\PYZus{}quantity\PYZus{}range = np.arange(70, 321, 5)   \PYZsh{} Doesn\PYZsq{}t do all combinations}

\PY{c+c1}{\PYZsh{} Method 1: Set property equal to an array}
\PY{n}{model\PYZus{}3}\PY{o}{.}\PY{n}{demand} \PY{o}{=} \PY{n}{demand\PYZus{}range}
\PY{n}{model\PYZus{}3}\PY{o}{.}\PY{n}{order\PYZus{}quantity} \PY{o}{=} \PY{n}{order\PYZus{}quantity\PYZus{}range}

\PY{c+c1}{\PYZsh{} model\PYZus{}3.profit()  \PYZsh{} This line will trigger the broadcast error}
\end{Verbatim}
\end{tcolorbox}

    Recall that for the non-OO model, we could easily do an n-way data table
using a list comprehension that made use of the
\texttt{bookstore\_profit} function because we could explicitly loop
over ranges of any subset of the input variables and call the
\texttt{bookstore\_profit} function with all of the input arguments
specified. We can't do that with the OO model because the
\texttt{profit} function is a method of the object and we don't pass it
any input arguments other than \texttt{self}. So, we need a way to loop
over all the input ranges and update the bookstore model object's input
attributes and then call the \texttt{profit} method.

What to do?

    \hypertarget{a-decision-point-in-our-software-design-process}{%
\subsubsection{A decision point in our software design
process}\label{a-decision-point-in-our-software-design-process}}

An end user modeler analagous to your basic Excel power user isn't going
to write OO code to do a Data Table. The list comprehension approach
based on a global function (i.e.~the non-OO model) is much more likely.
However, a relatively generic Data Table function could be implemented
in an OO way and be quite reusable. Also, we still want to implement
Goal Seek and it's not clear whether the OO or non-OO approach makes the
most sense.

Well, let's forge ahead and design and create a Python based Data Table
function that accepts an OO BookstoreModel object as one of its inputs.
Surely, we'll learn some things along the way.

    \hypertarget{adding-an-update-method}{%
\paragraph{\texorpdfstring{Adding an \texttt{update}
method}{Adding an update method}}\label{adding-an-update-method}}

Since were are going to need to repeatedly update some subset of the
input parameters, let's add an \texttt{update} method to the
BookstoreModel class that takes a dictionary of property, value pairs.
The question is, \textbf{how do we update object attributes from a
dictionary specifying the attribute and the new value?}

This StackoverFlow post was quite helpful:

https://stackoverflow.com/questions/2466191/set-attributes-from-dictionary-in-python

The idea is that we use Python's \texttt{setattr} function. Using
\href{https://www.w3schools.com/python/ref_func_setattr.asp}{\texttt{setattr}}
is really easy:

\begin{verbatim}
setattr(object, attribute name, attribute value)
\end{verbatim}

As a specific example, it can be used like this.

    \begin{tcolorbox}[breakable, size=fbox, boxrule=1pt, pad at break*=1mm,colback=cellbackground, colframe=cellborder]
\prompt{In}{incolor}{39}{\boxspacing}
\begin{Verbatim}[commandchars=\\\{\}]
\PY{n}{model\PYZus{}4} \PY{o}{=} \PY{n}{BookstoreModel}\PY{p}{(}\PY{n}{unit\PYZus{}cost}\PY{p}{,} \PY{n}{selling\PYZus{}price}\PY{p}{,} \PY{n}{unit\PYZus{}refund}\PY{p}{,} \PY{n}{order\PYZus{}quantity}\PY{p}{,} \PY{n}{demand}\PY{p}{)}
\PY{n+nb}{print}\PY{p}{(}\PY{n}{model\PYZus{}4}\PY{p}{)}

\PY{c+c1}{\PYZsh{} We want to update a existing model object with these parameter values}
\PY{n}{new\PYZus{}params} \PY{o}{=} \PY{p}{\PYZob{}}\PY{l+s+s1}{\PYZsq{}}\PY{l+s+s1}{unit\PYZus{}cost}\PY{l+s+s1}{\PYZsq{}}\PY{p}{:} \PY{l+m+mf}{8.5}\PY{p}{,} \PY{l+s+s1}{\PYZsq{}}\PY{l+s+s1}{order\PYZus{}quantity}\PY{l+s+s1}{\PYZsq{}}\PY{p}{:} \PY{l+m+mi}{250}\PY{p}{\PYZcb{}}

\PY{c+c1}{\PYZsh{} Iterate over the keys in new\PYZus{}params}
\PY{k}{for} \PY{n}{key} \PY{o+ow}{in} \PY{n}{new\PYZus{}params}\PY{p}{:}
            \PY{n+nb}{setattr}\PY{p}{(}\PY{n}{model\PYZus{}4}\PY{p}{,} \PY{n}{key}\PY{p}{,} \PY{n}{new\PYZus{}params}\PY{p}{[}\PY{n}{key}\PY{p}{]}\PY{p}{)}
        
\PY{n+nb}{print}\PY{p}{(}\PY{n}{model\PYZus{}4}\PY{p}{)}
\end{Verbatim}
\end{tcolorbox}

    \begin{Verbatim}[commandchars=\\\{\}]
\{'unit\_cost': 7.5, 'selling\_price': 10.0, 'unit\_refund': 2.5, 'order\_quantity':
200, 'demand': 193\}
\{'unit\_cost': 8.5, 'selling\_price': 10.0, 'unit\_refund': 2.5, 'order\_quantity':
250, 'demand': 193\}
    \end{Verbatim}

    Okay, here's our updated BookstoreModel class. Note that I've changed to
keyword arguments with default values.

    \begin{tcolorbox}[breakable, size=fbox, boxrule=1pt, pad at break*=1mm,colback=cellbackground, colframe=cellborder]
\prompt{In}{incolor}{40}{\boxspacing}
\begin{Verbatim}[commandchars=\\\{\}]
\PY{k}{class} \PY{n+nc}{BookstoreModel}\PY{p}{(}\PY{p}{)}\PY{p}{:}
    \PY{k}{def} \PY{n+nf+fm}{\PYZus{}\PYZus{}init\PYZus{}\PYZus{}}\PY{p}{(}\PY{n+nb+bp}{self}\PY{p}{,} \PY{n}{unit\PYZus{}cost}\PY{o}{=}\PY{l+m+mi}{0}\PY{p}{,} \PY{n}{selling\PYZus{}price}\PY{o}{=}\PY{l+m+mi}{0}\PY{p}{,} \PY{n}{unit\PYZus{}refund}\PY{o}{=}\PY{l+m+mi}{0}\PY{p}{,} 
                 \PY{n}{order\PYZus{}quantity}\PY{o}{=}\PY{l+m+mi}{0}\PY{p}{,} \PY{n}{demand}\PY{o}{=}\PY{l+m+mi}{0}\PY{p}{)}\PY{p}{:}
        \PY{n+nb+bp}{self}\PY{o}{.}\PY{n}{unit\PYZus{}cost} \PY{o}{=} \PY{n}{unit\PYZus{}cost}
        \PY{n+nb+bp}{self}\PY{o}{.}\PY{n}{selling\PYZus{}price} \PY{o}{=} \PY{n}{selling\PYZus{}price}
        \PY{n+nb+bp}{self}\PY{o}{.}\PY{n}{unit\PYZus{}refund} \PY{o}{=} \PY{n}{unit\PYZus{}refund}
        \PY{n+nb+bp}{self}\PY{o}{.}\PY{n}{order\PYZus{}quantity} \PY{o}{=} \PY{n}{order\PYZus{}quantity}
        \PY{n+nb+bp}{self}\PY{o}{.}\PY{n}{demand} \PY{o}{=} \PY{n}{demand}
        
        
    \PY{k}{def} \PY{n+nf}{update}\PY{p}{(}\PY{n+nb+bp}{self}\PY{p}{,} \PY{n}{param\PYZus{}dict}\PY{p}{)}\PY{p}{:}
        \PY{l+s+sd}{\PYZdq{}\PYZdq{}\PYZdq{}}
\PY{l+s+sd}{        Update parameter values}
\PY{l+s+sd}{        \PYZdq{}\PYZdq{}\PYZdq{}}
        \PY{k}{for} \PY{n}{key} \PY{o+ow}{in} \PY{n}{param\PYZus{}dict}\PY{p}{:}
            \PY{n+nb}{setattr}\PY{p}{(}\PY{n+nb+bp}{self}\PY{p}{,} \PY{n}{key}\PY{p}{,} \PY{n}{param\PYZus{}dict}\PY{p}{[}\PY{n}{key}\PY{p}{]}\PY{p}{)}
        
    \PY{k}{def} \PY{n+nf}{order\PYZus{}cost}\PY{p}{(}\PY{n+nb+bp}{self}\PY{p}{)}\PY{p}{:}
        \PY{l+s+sd}{\PYZdq{}\PYZdq{}\PYZdq{}Compute total order cost\PYZdq{}\PYZdq{}\PYZdq{}}
        \PY{k}{return} \PY{n+nb+bp}{self}\PY{o}{.}\PY{n}{unit\PYZus{}cost} \PY{o}{*} \PY{n+nb+bp}{self}\PY{o}{.}\PY{n}{order\PYZus{}quantity}
    
    \PY{k}{def} \PY{n+nf}{sales\PYZus{}revenue}\PY{p}{(}\PY{n+nb+bp}{self}\PY{p}{)}\PY{p}{:}
        \PY{l+s+sd}{\PYZdq{}\PYZdq{}\PYZdq{}Compute sales revenue\PYZdq{}\PYZdq{}\PYZdq{}}
        \PY{k}{return} \PY{n}{np}\PY{o}{.}\PY{n}{minimum}\PY{p}{(}\PY{n+nb+bp}{self}\PY{o}{.}\PY{n}{order\PYZus{}quantity}\PY{p}{,} \PY{n+nb+bp}{self}\PY{o}{.}\PY{n}{demand}\PY{p}{)} \PY{o}{*} \PY{n+nb+bp}{self}\PY{o}{.}\PY{n}{selling\PYZus{}price}
    
    \PY{k}{def} \PY{n+nf}{refund\PYZus{}revenue}\PY{p}{(}\PY{n+nb+bp}{self}\PY{p}{)}\PY{p}{:}
        \PY{l+s+sd}{\PYZdq{}\PYZdq{}\PYZdq{}Compute revenue from refunds for unsold items\PYZdq{}\PYZdq{}\PYZdq{}}
        \PY{k}{return} \PY{n}{np}\PY{o}{.}\PY{n}{maximum}\PY{p}{(}\PY{l+m+mi}{0}\PY{p}{,} \PY{n+nb+bp}{self}\PY{o}{.}\PY{n}{order\PYZus{}quantity} \PY{o}{\PYZhy{}} \PY{n+nb+bp}{self}\PY{o}{.}\PY{n}{demand}\PY{p}{)} \PY{o}{*} \PY{n+nb+bp}{self}\PY{o}{.}\PY{n}{unit\PYZus{}refund}
    
    \PY{k}{def} \PY{n+nf}{profit}\PY{p}{(}\PY{n+nb+bp}{self}\PY{p}{)}\PY{p}{:}
        \PY{l+s+sd}{\PYZsq{}\PYZsq{}\PYZsq{}}
\PY{l+s+sd}{        Compute profit in bookstore model}
\PY{l+s+sd}{        \PYZsq{}\PYZsq{}\PYZsq{}}
        \PY{n}{profit} \PY{o}{=} \PY{n+nb+bp}{self}\PY{o}{.}\PY{n}{sales\PYZus{}revenue}\PY{p}{(}\PY{p}{)} \PY{o}{+} \PY{n+nb+bp}{self}\PY{o}{.}\PY{n}{refund\PYZus{}revenue}\PY{p}{(}\PY{p}{)} \PY{o}{\PYZhy{}} \PY{n+nb+bp}{self}\PY{o}{.}\PY{n}{order\PYZus{}cost}\PY{p}{(}\PY{p}{)}
        \PY{k}{return} \PY{n}{profit}
       
    \PY{k}{def} \PY{n+nf+fm}{\PYZus{}\PYZus{}str\PYZus{}\PYZus{}}\PY{p}{(}\PY{n+nb+bp}{self}\PY{p}{)}\PY{p}{:}
        \PY{l+s+sd}{\PYZdq{}\PYZdq{}\PYZdq{}}
\PY{l+s+sd}{        Print dictionary of object attributes but don\PYZsq{}t include an underscore as first char}
\PY{l+s+sd}{        \PYZdq{}\PYZdq{}\PYZdq{}}
        \PY{k}{return} \PY{n+nb}{str}\PY{p}{(}\PY{n+nb}{vars}\PY{p}{(}\PY{n+nb+bp}{self}\PY{p}{)}\PY{p}{)}
        \PY{c+c1}{\PYZsh{}return str(\PYZob{}key: val for (key, val) in vars(self).items() if key[0] != \PYZsq{}\PYZus{}\PYZsq{}\PYZcb{})}
\end{Verbatim}
\end{tcolorbox}

    Let's try out our \texttt{update} method. Also, since now the
BookstoreModel class uses keyword arguments with defaults, I'm going to
create a new model object in a slightly different way:

\begin{itemize}
\tightlist
\item
  create a new model object and just take the zero default values for
  the input attributes
\item
  use the \texttt{update} method to set the input attributes via a
  dictionary
\end{itemize}

    \begin{tcolorbox}[breakable, size=fbox, boxrule=1pt, pad at break*=1mm,colback=cellbackground, colframe=cellborder]
\prompt{In}{incolor}{41}{\boxspacing}
\begin{Verbatim}[commandchars=\\\{\}]
\PY{n}{base\PYZus{}inputs} \PY{o}{=} \PY{p}{\PYZob{}}\PY{l+s+s1}{\PYZsq{}}\PY{l+s+s1}{unit\PYZus{}cost}\PY{l+s+s1}{\PYZsq{}}\PY{p}{:} \PY{l+m+mf}{7.50}\PY{p}{,}
              \PY{l+s+s1}{\PYZsq{}}\PY{l+s+s1}{selling\PYZus{}price}\PY{l+s+s1}{\PYZsq{}}\PY{p}{:} \PY{l+m+mf}{10.00}\PY{p}{,}
              \PY{l+s+s1}{\PYZsq{}}\PY{l+s+s1}{unit\PYZus{}refund}\PY{l+s+s1}{\PYZsq{}}\PY{p}{:} \PY{l+m+mf}{2.50}\PY{p}{,}
              \PY{l+s+s1}{\PYZsq{}}\PY{l+s+s1}{order\PYZus{}quantity}\PY{l+s+s1}{\PYZsq{}}\PY{p}{:} \PY{l+m+mi}{200}\PY{p}{,}
              \PY{l+s+s1}{\PYZsq{}}\PY{l+s+s1}{demand}\PY{l+s+s1}{\PYZsq{}}\PY{p}{:} \PY{l+m+mi}{193}\PY{p}{\PYZcb{}}

\PY{n}{model\PYZus{}5} \PY{o}{=} \PY{n}{BookstoreModel}\PY{p}{(}\PY{p}{)}
\PY{n}{model\PYZus{}5}\PY{o}{.}\PY{n}{update}\PY{p}{(}\PY{n}{base\PYZus{}inputs}\PY{p}{)}
\PY{n+nb}{print}\PY{p}{(}\PY{n}{model\PYZus{}5}\PY{p}{)}
\end{Verbatim}
\end{tcolorbox}

    \begin{Verbatim}[commandchars=\\\{\}]
\{'unit\_cost': 7.5, 'selling\_price': 10.0, 'unit\_refund': 2.5, 'order\_quantity':
200, 'demand': 193\}
    \end{Verbatim}

    \hypertarget{how-to-generate-and-iterate-over-scenarios-to-create-a-2-way-data-table}{%
\paragraph{How to generate and iterate over scenarios to create a 2-way
Data
Table?}\label{how-to-generate-and-iterate-over-scenarios-to-create-a-2-way-data-table}}

Let's regroup and remind ourselves where we are in the development
process for our 2-way data table function. We've got two input
variables, \texttt{demand} and \texttt{order\_quantity} and we'd like to
specify a range over which each variable should vary. Then, we want to
consider all combinations of these variable values and compute the
associated, say, profit (really, any ouput of our model). It would be
convenient to be be able to specify our input ranges with a dictionary
whose keys are the input variable names and the values are any valid
Python iterable (such as a list or the output of some function that
returns a range of values). For example:

    \begin{tcolorbox}[breakable, size=fbox, boxrule=1pt, pad at break*=1mm,colback=cellbackground, colframe=cellborder]
\prompt{In}{incolor}{42}{\boxspacing}
\begin{Verbatim}[commandchars=\\\{\}]
\PY{n}{dt\PYZus{}param\PYZus{}ranges} \PY{o}{=} \PY{p}{\PYZob{}}\PY{l+s+s1}{\PYZsq{}}\PY{l+s+s1}{demand}\PY{l+s+s1}{\PYZsq{}}\PY{p}{:} \PY{n}{np}\PY{o}{.}\PY{n}{arange}\PY{p}{(}\PY{l+m+mi}{70}\PY{p}{,} \PY{l+m+mi}{321}\PY{p}{,} \PY{l+m+mi}{25}\PY{p}{)}\PY{p}{,}
                   \PY{l+s+s1}{\PYZsq{}}\PY{l+s+s1}{order\PYZus{}quantity}\PY{l+s+s1}{\PYZsq{}}\PY{p}{:} \PY{n}{np}\PY{o}{.}\PY{n}{arange}\PY{p}{(}\PY{l+m+mi}{70}\PY{p}{,} \PY{l+m+mi}{321}\PY{p}{,} \PY{l+m+mi}{50}\PY{p}{)}\PY{p}{\PYZcb{}}
\end{Verbatim}
\end{tcolorbox}

    Our immediate goal is to convert this dictionary into a list of
dictionaries - one dictionary per scenario. Like this:

\begin{verbatim}
[{'demand': 70, 'order_quantity': 70},
 {'demand': 70, 'order_quantity': 120},
 {'demand': 70, 'order_quantity': 170},
 {'demand': 70, 'order_quantity': 220},
 {'demand': 70, 'order_quantity': 270},
 {'demand': 70, 'order_quantity': 320},
 {'demand': 95, 'order_quantity': 70},
 {'demand': 95, 'order_quantity': 120},
 ...
 ...
 ...
 {'demand': 320, 'order_quantity': 220},
 {'demand': 320, 'order_quantity': 270},
 {'demand': 320, 'order_quantity': 320}]
 
\end{verbatim}

Each of these component dictionaries is something we can use with our
new model object \texttt{update} method to set the corresponding values
for each scenario. Hmmm, feels like some clever iterating and zipping
and more iterating. Before doing that ourselves, can we do this using
the functionality of some existing Python library? Spoiler alert, yes,
we can.

    \hypertarget{learning-from-the-ideas-and-code-from-scikit-learn}{%
\paragraph{Learning from the ideas and code from
scikit-learn}\label{learning-from-the-ideas-and-code-from-scikit-learn}}

\href{https://scikit-learn.org/stable/index.html}{Scikit-learn} is an
extremely popular Python package for building, testing and using machine
learning models. In thinking about the input values for an n-way data
table, it feels similar to a
\href{https://scikit-learn.org/stable/modules/grid_search.html}{parameter
grid for a hyperparameter grid search in scikit learn}. Can we borrow
some of their patterns or code ideas?

https://scikit-learn.org/stable/modules/generated/sklearn.model\_selection.GridSearchCV.html\#sklearn.model\_selection.GridSearchCV

https://scikit-learn.org/stable/modules/grid\_search.html\#grid-search

Example from User Guide shows passing in two different grids to explore
as a list of dicts. Notice that each dict is like the setup for an Excel
Data Table - the key is just the name of some variable and the value is
some iterable. Don't worry about the details, just look at the structure
of \texttt{param\_grid}.

\begin{verbatim}
param_grid = [
  {'C': [1, 10, 100, 1000], 'kernel': ['linear']},
  {'C': [1, 10, 100, 1000], 'gamma': [0.001, 0.0001], 'kernel': ['rbf']},
 ]
 
\end{verbatim}

For example, the \texttt{GridSearchCV} takes an estimator object and a
parameter grid object. The parameter grid is a dict or list of dicts in
which the keys are the parameter names. SKLearn implements a
\texttt{ParameterGrid} class. Looking at its source, the docstring
contains the following example. Notice that the result of
\texttt{ParameterGrid(param\_grid)} is a list of dicts in which each
dict contains a specific combination of the parameter values, i.e., it
generates all the scenarios to explore.

\begin{verbatim}
Examples
--------
>>> from sklearn.model_selection import ParameterGrid
>>> param_grid = {'a': [1, 2], 'b': [True, False]}
>>> list(ParameterGrid(param_grid)) == (
...    [{'a': 1, 'b': True}, {'a': 1, 'b': False},
...     {'a': 2, 'b': True}, {'a': 2, 'b': False}])
True
\end{verbatim}

Hmm, we could use this class directly.

    \begin{tcolorbox}[breakable, size=fbox, boxrule=1pt, pad at break*=1mm,colback=cellbackground, colframe=cellborder]
\prompt{In}{incolor}{43}{\boxspacing}
\begin{Verbatim}[commandchars=\\\{\}]
\PY{k+kn}{from} \PY{n+nn}{sklearn}\PY{n+nn}{.}\PY{n+nn}{model\PYZus{}selection}\PY{n+nn}{.}\PY{n+nn}{\PYZus{}search} \PY{k+kn}{import} \PY{n}{ParameterGrid}
\end{Verbatim}
\end{tcolorbox}

    \begin{tcolorbox}[breakable, size=fbox, boxrule=1pt, pad at break*=1mm,colback=cellbackground, colframe=cellborder]
\prompt{In}{incolor}{44}{\boxspacing}
\begin{Verbatim}[commandchars=\\\{\}]
\PY{n}{dt\PYZus{}param\PYZus{}ranges} \PY{o}{=} \PY{p}{\PYZob{}}\PY{l+s+s1}{\PYZsq{}}\PY{l+s+s1}{demand}\PY{l+s+s1}{\PYZsq{}}\PY{p}{:} \PY{n}{np}\PY{o}{.}\PY{n}{arange}\PY{p}{(}\PY{l+m+mi}{70}\PY{p}{,} \PY{l+m+mi}{321}\PY{p}{,} \PY{l+m+mi}{25}\PY{p}{)}\PY{p}{,}
                   \PY{l+s+s1}{\PYZsq{}}\PY{l+s+s1}{order\PYZus{}quantity}\PY{l+s+s1}{\PYZsq{}}\PY{p}{:} \PY{n}{np}\PY{o}{.}\PY{n}{arange}\PY{p}{(}\PY{l+m+mi}{70}\PY{p}{,} \PY{l+m+mi}{321}\PY{p}{,} \PY{l+m+mi}{50}\PY{p}{)}\PY{p}{\PYZcb{}}
\end{Verbatim}
\end{tcolorbox}

    \begin{tcolorbox}[breakable, size=fbox, boxrule=1pt, pad at break*=1mm,colback=cellbackground, colframe=cellborder]
\prompt{In}{incolor}{45}{\boxspacing}
\begin{Verbatim}[commandchars=\\\{\}]
\PY{n}{dt\PYZus{}param\PYZus{}grid} \PY{o}{=} \PY{n+nb}{list}\PY{p}{(}\PY{n}{ParameterGrid}\PY{p}{(}\PY{n}{dt\PYZus{}param\PYZus{}ranges}\PY{p}{)}\PY{p}{)}
\PY{n+nb}{print}\PY{p}{(}\PY{n}{dt\PYZus{}param\PYZus{}grid}\PY{p}{)}
\end{Verbatim}
\end{tcolorbox}

    \begin{Verbatim}[commandchars=\\\{\}]
[\{'demand': 70, 'order\_quantity': 70\}, \{'demand': 70, 'order\_quantity': 120\},
\{'demand': 70, 'order\_quantity': 170\}, \{'demand': 70, 'order\_quantity': 220\},
\{'demand': 70, 'order\_quantity': 270\}, \{'demand': 70, 'order\_quantity': 320\},
\{'demand': 95, 'order\_quantity': 70\}, \{'demand': 95, 'order\_quantity': 120\},
\{'demand': 95, 'order\_quantity': 170\}, \{'demand': 95, 'order\_quantity': 220\},
\{'demand': 95, 'order\_quantity': 270\}, \{'demand': 95, 'order\_quantity': 320\},
\{'demand': 120, 'order\_quantity': 70\}, \{'demand': 120, 'order\_quantity': 120\},
\{'demand': 120, 'order\_quantity': 170\}, \{'demand': 120, 'order\_quantity': 220\},
\{'demand': 120, 'order\_quantity': 270\}, \{'demand': 120, 'order\_quantity': 320\},
\{'demand': 145, 'order\_quantity': 70\}, \{'demand': 145, 'order\_quantity': 120\},
\{'demand': 145, 'order\_quantity': 170\}, \{'demand': 145, 'order\_quantity': 220\},
\{'demand': 145, 'order\_quantity': 270\}, \{'demand': 145, 'order\_quantity': 320\},
\{'demand': 170, 'order\_quantity': 70\}, \{'demand': 170, 'order\_quantity': 120\},
\{'demand': 170, 'order\_quantity': 170\}, \{'demand': 170, 'order\_quantity': 220\},
\{'demand': 170, 'order\_quantity': 270\}, \{'demand': 170, 'order\_quantity': 320\},
\{'demand': 195, 'order\_quantity': 70\}, \{'demand': 195, 'order\_quantity': 120\},
\{'demand': 195, 'order\_quantity': 170\}, \{'demand': 195, 'order\_quantity': 220\},
\{'demand': 195, 'order\_quantity': 270\}, \{'demand': 195, 'order\_quantity': 320\},
\{'demand': 220, 'order\_quantity': 70\}, \{'demand': 220, 'order\_quantity': 120\},
\{'demand': 220, 'order\_quantity': 170\}, \{'demand': 220, 'order\_quantity': 220\},
\{'demand': 220, 'order\_quantity': 270\}, \{'demand': 220, 'order\_quantity': 320\},
\{'demand': 245, 'order\_quantity': 70\}, \{'demand': 245, 'order\_quantity': 120\},
\{'demand': 245, 'order\_quantity': 170\}, \{'demand': 245, 'order\_quantity': 220\},
\{'demand': 245, 'order\_quantity': 270\}, \{'demand': 245, 'order\_quantity': 320\},
\{'demand': 270, 'order\_quantity': 70\}, \{'demand': 270, 'order\_quantity': 120\},
\{'demand': 270, 'order\_quantity': 170\}, \{'demand': 270, 'order\_quantity': 220\},
\{'demand': 270, 'order\_quantity': 270\}, \{'demand': 270, 'order\_quantity': 320\},
\{'demand': 295, 'order\_quantity': 70\}, \{'demand': 295, 'order\_quantity': 120\},
\{'demand': 295, 'order\_quantity': 170\}, \{'demand': 295, 'order\_quantity': 220\},
\{'demand': 295, 'order\_quantity': 270\}, \{'demand': 295, 'order\_quantity': 320\},
\{'demand': 320, 'order\_quantity': 70\}, \{'demand': 320, 'order\_quantity': 120\},
\{'demand': 320, 'order\_quantity': 170\}, \{'demand': 320, 'order\_quantity': 220\},
\{'demand': 320, 'order\_quantity': 270\}, \{'demand': 320, 'order\_quantity': 320\}]
    \end{Verbatim}

    Perfect! We have a list of dictionaries, each of which contains one of
the combinations of the \texttt{demand} and \texttt{order\_quantity}
attributes that we want in our data table. Isn't it nice that we just
implemented an \texttt{update} method that takes just such a dictionary
as its input. Almost like we planned it. :)

While there's nothing wrong with leveraging the skikit-learn library for
this task (after all, reusing objects is one of the whole reasons for
OOP), we really don't need all the generality provided by its
\texttt{ParameterGrid} class. We just need to do some careful iteration
through our dictionary of input ranges to create the same output. It
will give us a chance to use the very useful
\href{https://docs.python.org/3/library/itertools.html}{itertools
library} and see Python's \texttt{zip} function in action.

    \begin{tcolorbox}[breakable, size=fbox, boxrule=1pt, pad at break*=1mm,colback=cellbackground, colframe=cellborder]
\prompt{In}{incolor}{46}{\boxspacing}
\begin{Verbatim}[commandchars=\\\{\}]
\PY{k+kn}{from} \PY{n+nn}{itertools} \PY{k+kn}{import} \PY{n}{product}
\end{Verbatim}
\end{tcolorbox}

    \begin{tcolorbox}[breakable, size=fbox, boxrule=1pt, pad at break*=1mm,colback=cellbackground, colframe=cellborder]
\prompt{In}{incolor}{47}{\boxspacing}
\begin{Verbatim}[commandchars=\\\{\}]
\PY{c+c1}{\PYZsh{} Look at the key, value pairs (tuples) in dt\PYZus{}param\PYZus{}ranges dictionary}
\PY{n}{dt\PYZus{}param\PYZus{}ranges}\PY{o}{.}\PY{n}{items}\PY{p}{(}\PY{p}{)}
\end{Verbatim}
\end{tcolorbox}

            \begin{tcolorbox}[breakable, size=fbox, boxrule=.5pt, pad at break*=1mm, opacityfill=0]
\prompt{Out}{outcolor}{47}{\boxspacing}
\begin{Verbatim}[commandchars=\\\{\}]
dict\_items([('demand', array([ 70,  95, 120, 145, 170, 195, 220, 245, 270, 295,
320])), ('order\_quantity', array([ 70, 120, 170, 220, 270, 320]))])
\end{Verbatim}
\end{tcolorbox}
        
    Now we can use
\href{https://docs.python.org/3.6/tutorial/controlflow.html\#unpacking-argument-lists}{unpacking}
along with the
\href{https://www.w3schools.com/python/ref_func_zip.asp}{zip function}
to get two tuples. One tuple will get stored in the variable
\texttt{keys} and the other in \texttt{values}.

    \begin{tcolorbox}[breakable, size=fbox, boxrule=1pt, pad at break*=1mm,colback=cellbackground, colframe=cellborder]
\prompt{In}{incolor}{48}{\boxspacing}
\begin{Verbatim}[commandchars=\\\{\}]
\PY{n+nb}{print}\PY{p}{(}\PY{l+s+s2}{\PYZdq{}}\PY{l+s+s2}{The original tuples}\PY{l+s+s2}{\PYZdq{}}\PY{p}{)}
\PY{n+nb}{print}\PY{p}{(}\PY{o}{*}\PY{n}{dt\PYZus{}param\PYZus{}ranges}\PY{o}{.}\PY{n}{items}\PY{p}{(}\PY{p}{)}\PY{p}{)}
\PY{n+nb}{print}\PY{p}{(}\PY{l+s+s2}{\PYZdq{}}\PY{l+s+se}{\PYZbs{}n}\PY{l+s+s2}{The zipped tuples}\PY{l+s+s2}{\PYZdq{}}\PY{p}{)}
\PY{n}{keys}\PY{p}{,} \PY{n}{values} \PY{o}{=} \PY{n+nb}{zip}\PY{p}{(}\PY{o}{*}\PY{n}{dt\PYZus{}param\PYZus{}ranges}\PY{o}{.}\PY{n}{items}\PY{p}{(}\PY{p}{)}\PY{p}{)}
\PY{n+nb}{print}\PY{p}{(}\PY{n}{keys}\PY{p}{,} \PY{n}{values}\PY{p}{)}
\end{Verbatim}
\end{tcolorbox}

    \begin{Verbatim}[commandchars=\\\{\}]
The original tuples
('demand', array([ 70,  95, 120, 145, 170, 195, 220, 245, 270, 295, 320]))
('order\_quantity', array([ 70, 120, 170, 220, 270, 320]))

The zipped tuples
('demand', 'order\_quantity') (array([ 70,  95, 120, 145, 170, 195, 220, 245,
270, 295, 320]), array([ 70, 120, 170, 220, 270, 320]))
    \end{Verbatim}

    Now we have a tuple of the keys,
\texttt{(\textquotesingle{}demand\textquotesingle{},\ \textquotesingle{}order\_quantity\textquotesingle{})}
and a tuple containing the two arrays of associated values.

What we need next is a way to get all the combinations of the values in
the two arrays. This is a perfect job for the \texttt{itertools.product}
function. Here's a simple example of what \texttt{product} does.

    \begin{tcolorbox}[breakable, size=fbox, boxrule=1pt, pad at break*=1mm,colback=cellbackground, colframe=cellborder]
\prompt{In}{incolor}{49}{\boxspacing}
\begin{Verbatim}[commandchars=\\\{\}]
\PY{n}{colors} \PY{o}{=} \PY{p}{[}\PY{l+s+s1}{\PYZsq{}}\PY{l+s+s1}{red}\PY{l+s+s1}{\PYZsq{}}\PY{p}{,} \PY{l+s+s1}{\PYZsq{}}\PY{l+s+s1}{blue}\PY{l+s+s1}{\PYZsq{}}\PY{p}{,} \PY{l+s+s1}{\PYZsq{}}\PY{l+s+s1}{green}\PY{l+s+s1}{\PYZsq{}}\PY{p}{]}
\PY{n}{intensities} \PY{o}{=} \PY{p}{[}\PY{l+s+s1}{\PYZsq{}}\PY{l+s+s1}{pale}\PY{l+s+s1}{\PYZsq{}}\PY{p}{,} \PY{l+s+s1}{\PYZsq{}}\PY{l+s+s1}{bright}\PY{l+s+s1}{\PYZsq{}}\PY{p}{]}
\PY{k}{for} \PY{n}{pair} \PY{o+ow}{in} \PY{n}{product}\PY{p}{(}\PY{n}{intensities}\PY{p}{,} \PY{n}{colors}\PY{p}{)}\PY{p}{:}
    \PY{n+nb}{print}\PY{p}{(}\PY{n}{pair}\PY{p}{)}
\end{Verbatim}
\end{tcolorbox}

    \begin{Verbatim}[commandchars=\\\{\}]
('pale', 'red')
('pale', 'blue')
('pale', 'green')
('bright', 'red')
('bright', 'blue')
('bright', 'green')
    \end{Verbatim}

    The \texttt{values} variable is a tuple (that needs unpacking)
containing the two arrays for which we want the \texttt{product}. I'll
turn the result into a list so we can see it. Try it without the list to
see what you get.

    \begin{tcolorbox}[breakable, size=fbox, boxrule=1pt, pad at break*=1mm,colback=cellbackground, colframe=cellborder]
\prompt{In}{incolor}{50}{\boxspacing}
\begin{Verbatim}[commandchars=\\\{\}]
\PY{n+nb}{print}\PY{p}{(}\PY{n}{values}\PY{p}{)}
\end{Verbatim}
\end{tcolorbox}

    \begin{Verbatim}[commandchars=\\\{\}]
(array([ 70,  95, 120, 145, 170, 195, 220, 245, 270, 295, 320]), array([ 70,
120, 170, 220, 270, 320]))
    \end{Verbatim}

    \begin{tcolorbox}[breakable, size=fbox, boxrule=1pt, pad at break*=1mm,colback=cellbackground, colframe=cellborder]
\prompt{In}{incolor}{51}{\boxspacing}
\begin{Verbatim}[commandchars=\\\{\}]
\PY{n+nb}{print}\PY{p}{(}\PY{n+nb}{list}\PY{p}{(}\PY{n}{product}\PY{p}{(}\PY{o}{*}\PY{n}{values}\PY{p}{)}\PY{p}{)}\PY{p}{)}
\end{Verbatim}
\end{tcolorbox}

    \begin{Verbatim}[commandchars=\\\{\}]
[(70, 70), (70, 120), (70, 170), (70, 220), (70, 270), (70, 320), (95, 70), (95,
120), (95, 170), (95, 220), (95, 270), (95, 320), (120, 70), (120, 120), (120,
170), (120, 220), (120, 270), (120, 320), (145, 70), (145, 120), (145, 170),
(145, 220), (145, 270), (145, 320), (170, 70), (170, 120), (170, 170), (170,
220), (170, 270), (170, 320), (195, 70), (195, 120), (195, 170), (195, 220),
(195, 270), (195, 320), (220, 70), (220, 120), (220, 170), (220, 220), (220,
270), (220, 320), (245, 70), (245, 120), (245, 170), (245, 220), (245, 270),
(245, 320), (270, 70), (270, 120), (270, 170), (270, 220), (270, 270), (270,
320), (295, 70), (295, 120), (295, 170), (295, 220), (295, 270), (295, 320),
(320, 70), (320, 120), (320, 170), (320, 220), (320, 270), (320, 320)]
    \end{Verbatim}

    Finally, we need to zip each of these tuples with the \texttt{keys}
variable and turn the resulting tuple into a dictionary. Let's recall
what it contains:

    \begin{tcolorbox}[breakable, size=fbox, boxrule=1pt, pad at break*=1mm,colback=cellbackground, colframe=cellborder]
\prompt{In}{incolor}{52}{\boxspacing}
\begin{Verbatim}[commandchars=\\\{\}]
\PY{n}{keys}
\end{Verbatim}
\end{tcolorbox}

            \begin{tcolorbox}[breakable, size=fbox, boxrule=.5pt, pad at break*=1mm, opacityfill=0]
\prompt{Out}{outcolor}{52}{\boxspacing}
\begin{Verbatim}[commandchars=\\\{\}]
('demand', 'order\_quantity')
\end{Verbatim}
\end{tcolorbox}
        
    The \texttt{zip} function works like a zipper and pairs elements in the
same position in our two tuples.

    \begin{tcolorbox}[breakable, size=fbox, boxrule=1pt, pad at break*=1mm,colback=cellbackground, colframe=cellborder]
\prompt{In}{incolor}{53}{\boxspacing}
\begin{Verbatim}[commandchars=\\\{\}]
\PY{n+nb}{list}\PY{p}{(}\PY{n+nb}{zip}\PY{p}{(}\PY{n}{keys}\PY{p}{,} \PY{p}{(}\PY{l+m+mi}{70}\PY{p}{,} \PY{l+m+mi}{120}\PY{p}{)}\PY{p}{)}\PY{p}{)}
\end{Verbatim}
\end{tcolorbox}

            \begin{tcolorbox}[breakable, size=fbox, boxrule=.5pt, pad at break*=1mm, opacityfill=0]
\prompt{Out}{outcolor}{53}{\boxspacing}
\begin{Verbatim}[commandchars=\\\{\}]
[('demand', 70), ('order\_quantity', 120)]
\end{Verbatim}
\end{tcolorbox}
        
    \ldots{} and it's easy to turn this into a dictionary (which is what we
want).

    \begin{tcolorbox}[breakable, size=fbox, boxrule=1pt, pad at break*=1mm,colback=cellbackground, colframe=cellborder]
\prompt{In}{incolor}{54}{\boxspacing}
\begin{Verbatim}[commandchars=\\\{\}]
\PY{n+nb}{dict}\PY{p}{(}\PY{n+nb}{zip}\PY{p}{(}\PY{n}{keys}\PY{p}{,} \PY{p}{(}\PY{l+m+mi}{70}\PY{p}{,} \PY{l+m+mi}{120}\PY{p}{)}\PY{p}{)}\PY{p}{)}
\end{Verbatim}
\end{tcolorbox}

            \begin{tcolorbox}[breakable, size=fbox, boxrule=.5pt, pad at break*=1mm, opacityfill=0]
\prompt{Out}{outcolor}{54}{\boxspacing}
\begin{Verbatim}[commandchars=\\\{\}]
\{'demand': 70, 'order\_quantity': 120\}
\end{Verbatim}
\end{tcolorbox}
        
    Ok, just put all these ideas together.

    \begin{tcolorbox}[breakable, size=fbox, boxrule=1pt, pad at break*=1mm,colback=cellbackground, colframe=cellborder]
\prompt{In}{incolor}{55}{\boxspacing}
\begin{Verbatim}[commandchars=\\\{\}]
\PY{n}{param\PYZus{}grid} \PY{o}{=} \PY{p}{[}\PY{p}{]}

\PY{n}{keys}\PY{p}{,} \PY{n}{values} \PY{o}{=} \PY{n+nb}{zip}\PY{p}{(}\PY{o}{*}\PY{n}{dt\PYZus{}param\PYZus{}ranges}\PY{o}{.}\PY{n}{items}\PY{p}{(}\PY{p}{)}\PY{p}{)}

\PY{k}{for} \PY{n}{scenario} \PY{o+ow}{in} \PY{n}{product}\PY{p}{(}\PY{o}{*}\PY{n}{values}\PY{p}{)}\PY{p}{:}
    \PY{n}{params} \PY{o}{=} \PY{n+nb}{dict}\PY{p}{(}\PY{n+nb}{zip}\PY{p}{(}\PY{n}{keys}\PY{p}{,} \PY{n}{scenario}\PY{p}{)}\PY{p}{)}
    \PY{n}{param\PYZus{}grid}\PY{o}{.}\PY{n}{append}\PY{p}{(}\PY{n}{params}\PY{p}{)}
        
\PY{n+nb}{print}\PY{p}{(}\PY{n}{param\PYZus{}grid}\PY{p}{)}
\end{Verbatim}
\end{tcolorbox}

    \begin{Verbatim}[commandchars=\\\{\}]
[\{'demand': 70, 'order\_quantity': 70\}, \{'demand': 70, 'order\_quantity': 120\},
\{'demand': 70, 'order\_quantity': 170\}, \{'demand': 70, 'order\_quantity': 220\},
\{'demand': 70, 'order\_quantity': 270\}, \{'demand': 70, 'order\_quantity': 320\},
\{'demand': 95, 'order\_quantity': 70\}, \{'demand': 95, 'order\_quantity': 120\},
\{'demand': 95, 'order\_quantity': 170\}, \{'demand': 95, 'order\_quantity': 220\},
\{'demand': 95, 'order\_quantity': 270\}, \{'demand': 95, 'order\_quantity': 320\},
\{'demand': 120, 'order\_quantity': 70\}, \{'demand': 120, 'order\_quantity': 120\},
\{'demand': 120, 'order\_quantity': 170\}, \{'demand': 120, 'order\_quantity': 220\},
\{'demand': 120, 'order\_quantity': 270\}, \{'demand': 120, 'order\_quantity': 320\},
\{'demand': 145, 'order\_quantity': 70\}, \{'demand': 145, 'order\_quantity': 120\},
\{'demand': 145, 'order\_quantity': 170\}, \{'demand': 145, 'order\_quantity': 220\},
\{'demand': 145, 'order\_quantity': 270\}, \{'demand': 145, 'order\_quantity': 320\},
\{'demand': 170, 'order\_quantity': 70\}, \{'demand': 170, 'order\_quantity': 120\},
\{'demand': 170, 'order\_quantity': 170\}, \{'demand': 170, 'order\_quantity': 220\},
\{'demand': 170, 'order\_quantity': 270\}, \{'demand': 170, 'order\_quantity': 320\},
\{'demand': 195, 'order\_quantity': 70\}, \{'demand': 195, 'order\_quantity': 120\},
\{'demand': 195, 'order\_quantity': 170\}, \{'demand': 195, 'order\_quantity': 220\},
\{'demand': 195, 'order\_quantity': 270\}, \{'demand': 195, 'order\_quantity': 320\},
\{'demand': 220, 'order\_quantity': 70\}, \{'demand': 220, 'order\_quantity': 120\},
\{'demand': 220, 'order\_quantity': 170\}, \{'demand': 220, 'order\_quantity': 220\},
\{'demand': 220, 'order\_quantity': 270\}, \{'demand': 220, 'order\_quantity': 320\},
\{'demand': 245, 'order\_quantity': 70\}, \{'demand': 245, 'order\_quantity': 120\},
\{'demand': 245, 'order\_quantity': 170\}, \{'demand': 245, 'order\_quantity': 220\},
\{'demand': 245, 'order\_quantity': 270\}, \{'demand': 245, 'order\_quantity': 320\},
\{'demand': 270, 'order\_quantity': 70\}, \{'demand': 270, 'order\_quantity': 120\},
\{'demand': 270, 'order\_quantity': 170\}, \{'demand': 270, 'order\_quantity': 220\},
\{'demand': 270, 'order\_quantity': 270\}, \{'demand': 270, 'order\_quantity': 320\},
\{'demand': 295, 'order\_quantity': 70\}, \{'demand': 295, 'order\_quantity': 120\},
\{'demand': 295, 'order\_quantity': 170\}, \{'demand': 295, 'order\_quantity': 220\},
\{'demand': 295, 'order\_quantity': 270\}, \{'demand': 295, 'order\_quantity': 320\},
\{'demand': 320, 'order\_quantity': 70\}, \{'demand': 320, 'order\_quantity': 120\},
\{'demand': 320, 'order\_quantity': 170\}, \{'demand': 320, 'order\_quantity': 220\},
\{'demand': 320, 'order\_quantity': 270\}, \{'demand': 320, 'order\_quantity': 320\}]
    \end{Verbatim}

    Voila! We've recreated the same output as \texttt{ParameterGrid} gave
us.

\begin{quote}
This little exercise underscores the value of knowing the basics of OOP
if you are going to do analytics work in Python. Almost all of the main
libraries that you'll encounter (e.g.~pandas, matplotlib, scikit-learn)
are written in OO fashion. So, in order for you to be able to peek into
their source code, make sense of things, and maybe even borrow some code
as we did, you need to understand the basics of OOP in Python.
\end{quote}

    \hypertarget{creating-the-data_table-function}{%
\subsubsection{Creating the data\_table
function}\label{creating-the-data_table-function}}

Our plan is to create a function that takes three inputs:

\begin{itemize}
\tightlist
\item
  a model object such as \texttt{model\_5},
\item
  a dictionary containing the input attributes and associated ranges for
  the data table such as \texttt{dt\_param\_ranges},
\item
  a list of outputs which are methods of the the model object; for
  example \texttt{{[}\textquotesingle{}profit\textquotesingle{}{]}}
\end{itemize}

Since we don't want the \texttt{data\_table} function to modify our
passed in model object, we'll create a copy of the model within the
function. As we saw when copying list objects, we need to be careful
when copying user defined objects. We'll create what is known as a
\emph{deep copy} using the \texttt{copy} library. Learn more about
\href{https://docs.python.org/3/library/copy.html}{shallow and deep
copying here}.

Then we'll generate our parameter grid. I like to think of each element
in the parameter grid,
e.g.~\texttt{\{\textquotesingle{}demand\textquotesingle{}:\ 295,\ \textquotesingle{}order\_quantity\textquotesingle{}:\ 270\}},
as a \emph{scenario}. We loop over all the scenarios in the grid,
updating the model copy and computing the requested outputs. For each
scenario, we'll end up with a dictionary containing both the inputs and
outputs. For example,
\texttt{\{\textquotesingle{}demand\textquotesingle{}:\ 295,\ \textquotesingle{}order\_quantity\textquotesingle{}:\ 270,\ \textquotesingle{}profit\textquotesingle{}:405.0\}}.
Each such dictionary will get stored in a list and then at the very end
our function will convert this list to a pandas DataFrame and return it.

Here's our first version. Take note of the use of \texttt{getattr} to
indirectly call methods (e.g.~\texttt{profit}) of the model.

    \begin{tcolorbox}[breakable, size=fbox, boxrule=1pt, pad at break*=1mm,colback=cellbackground, colframe=cellborder]
\prompt{In}{incolor}{56}{\boxspacing}
\begin{Verbatim}[commandchars=\\\{\}]
\PY{c+c1}{\PYZsh{} Demo of getattr for indirect method calling}
\PY{n+nb}{getattr}\PY{p}{(}\PY{n}{model\PYZus{}5}\PY{p}{,} \PY{l+s+s1}{\PYZsq{}}\PY{l+s+s1}{profit}\PY{l+s+s1}{\PYZsq{}}\PY{p}{)}\PY{p}{(}\PY{p}{)}
\end{Verbatim}
\end{tcolorbox}

            \begin{tcolorbox}[breakable, size=fbox, boxrule=.5pt, pad at break*=1mm, opacityfill=0]
\prompt{Out}{outcolor}{56}{\boxspacing}
\begin{Verbatim}[commandchars=\\\{\}]
447.5
\end{Verbatim}
\end{tcolorbox}
        
    \begin{tcolorbox}[breakable, size=fbox, boxrule=1pt, pad at break*=1mm,colback=cellbackground, colframe=cellborder]
\prompt{In}{incolor}{57}{\boxspacing}
\begin{Verbatim}[commandchars=\\\{\}]
\PY{k}{def} \PY{n+nf}{data\PYZus{}table}\PY{p}{(}\PY{n}{model}\PY{p}{,} \PY{n}{scenario\PYZus{}inputs}\PY{p}{,} \PY{n}{outputs}\PY{p}{)}\PY{p}{:}
    \PY{l+s+sd}{\PYZsq{}\PYZsq{}\PYZsq{}Create n\PYZhy{}inputs by m\PYZhy{}outputs data table. }

\PY{l+s+sd}{    Parameters}
\PY{l+s+sd}{    \PYZhy{}\PYZhy{}\PYZhy{}\PYZhy{}\PYZhy{}\PYZhy{}\PYZhy{}\PYZhy{}\PYZhy{}\PYZhy{}}
\PY{l+s+sd}{    model : object}
\PY{l+s+sd}{        User defined object containing the appropriate methods and properties for computing outputs from inputs}
\PY{l+s+sd}{    scenario\PYZus{}inputs : dict of str to sequence}
\PY{l+s+sd}{        Keys are input variable names and values are sequence of values for each scenario for this variable.}
\PY{l+s+sd}{    outputs : list of str}
\PY{l+s+sd}{        List of output variable names}

\PY{l+s+sd}{    Returns}
\PY{l+s+sd}{    \PYZhy{}\PYZhy{}\PYZhy{}\PYZhy{}\PYZhy{}\PYZhy{}\PYZhy{}}
\PY{l+s+sd}{    results\PYZus{}df : pandas DataFrame}
\PY{l+s+sd}{        Contains values of all outputs for every combination of scenario inputs}
\PY{l+s+sd}{    \PYZsq{}\PYZsq{}\PYZsq{}}

    \PY{c+c1}{\PYZsh{} Clone the model using deepcopy}
    \PY{n}{model\PYZus{}clone} \PY{o}{=} \PY{n}{copy}\PY{o}{.}\PY{n}{deepcopy}\PY{p}{(}\PY{n}{model}\PY{p}{)}
    
    \PY{c+c1}{\PYZsh{} Create parameter grid}
    \PY{n}{dt\PYZus{}param\PYZus{}grid} \PY{o}{=} \PY{n+nb}{list}\PY{p}{(}\PY{n}{ParameterGrid}\PY{p}{(}\PY{n}{scenario\PYZus{}inputs}\PY{p}{)}\PY{p}{)}
    
    \PY{c+c1}{\PYZsh{} Create the table as a list of dictionaries}
    \PY{n}{results} \PY{o}{=} \PY{p}{[}\PY{p}{]}

    \PY{c+c1}{\PYZsh{} Loop over the scenarios}
    \PY{k}{for} \PY{n}{params} \PY{o+ow}{in} \PY{n}{dt\PYZus{}param\PYZus{}grid}\PY{p}{:}
        \PY{c+c1}{\PYZsh{} Update the model clone with scenario specific values}
        \PY{n}{model\PYZus{}clone}\PY{o}{.}\PY{n}{update}\PY{p}{(}\PY{n}{params}\PY{p}{)}
        \PY{c+c1}{\PYZsh{} Create a result dictionary based on a copy of the scenario inputs}
        \PY{n}{result} \PY{o}{=} \PY{n}{copy}\PY{o}{.}\PY{n}{copy}\PY{p}{(}\PY{n}{params}\PY{p}{)}
        \PY{c+c1}{\PYZsh{} Loop over the list of requested outputs}
        \PY{k}{for} \PY{n}{output} \PY{o+ow}{in} \PY{n}{outputs}\PY{p}{:}
            \PY{c+c1}{\PYZsh{} Compute the output.}
            \PY{n}{out\PYZus{}val} \PY{o}{=} \PY{n+nb}{getattr}\PY{p}{(}\PY{n}{model\PYZus{}clone}\PY{p}{,} \PY{n}{output}\PY{p}{)}\PY{p}{(}\PY{p}{)}
            \PY{c+c1}{\PYZsh{} Add the output to the result dictionary}
            \PY{n}{result}\PY{p}{[}\PY{n}{output}\PY{p}{]} \PY{o}{=} \PY{n}{out\PYZus{}val}
        
        \PY{c+c1}{\PYZsh{} Append the result dictionary to the results list}
        \PY{n}{results}\PY{o}{.}\PY{n}{append}\PY{p}{(}\PY{n}{result}\PY{p}{)}

    \PY{c+c1}{\PYZsh{} Convert the results list (of dictionaries) to a pandas DataFrame and return it}
    \PY{n}{results\PYZus{}df} \PY{o}{=} \PY{n}{pd}\PY{o}{.}\PY{n}{DataFrame}\PY{p}{(}\PY{n}{results}\PY{p}{)}
    \PY{k}{return} \PY{n}{results\PYZus{}df}
\end{Verbatim}
\end{tcolorbox}

    Okay, let's try it out. This time, I'll use \emph{dictionary unpacking}
to pass in the parameter values to our class.

\begin{itemize}
\tightlist
\item
  Create a dict called \texttt{base\_inputs} with values to use in the
  model
\item
  Then \textbf{base\_inputs} will \emph{unpack} the dictionary to key,
  value pairs that are used to specificy the keyword input arguments for
  out \texttt{BookstoreModel}.
\end{itemize}

    \begin{tcolorbox}[breakable, size=fbox, boxrule=1pt, pad at break*=1mm,colback=cellbackground, colframe=cellborder]
\prompt{In}{incolor}{58}{\boxspacing}
\begin{Verbatim}[commandchars=\\\{\}]
\PY{c+c1}{\PYZsh{} Create a dictionary of base input values}

\PY{n}{base\PYZus{}inputs} \PY{o}{=} \PY{p}{\PYZob{}}\PY{l+s+s1}{\PYZsq{}}\PY{l+s+s1}{unit\PYZus{}cost}\PY{l+s+s1}{\PYZsq{}}\PY{p}{:} \PY{l+m+mf}{7.5}\PY{p}{,} 
               \PY{l+s+s1}{\PYZsq{}}\PY{l+s+s1}{selling\PYZus{}price}\PY{l+s+s1}{\PYZsq{}}\PY{p}{:} \PY{l+m+mf}{10.0}\PY{p}{,} 
               \PY{l+s+s1}{\PYZsq{}}\PY{l+s+s1}{unit\PYZus{}refund}\PY{l+s+s1}{\PYZsq{}}\PY{p}{:} \PY{l+m+mf}{2.5}\PY{p}{,} 
               \PY{l+s+s1}{\PYZsq{}}\PY{l+s+s1}{order\PYZus{}quantity}\PY{l+s+s1}{\PYZsq{}}\PY{p}{:} \PY{l+m+mi}{200}\PY{p}{,} 
               \PY{l+s+s1}{\PYZsq{}}\PY{l+s+s1}{demand}\PY{l+s+s1}{\PYZsq{}}\PY{p}{:} \PY{l+m+mi}{193}\PY{p}{\PYZcb{}}
\end{Verbatim}
\end{tcolorbox}

    \begin{tcolorbox}[breakable, size=fbox, boxrule=1pt, pad at break*=1mm,colback=cellbackground, colframe=cellborder]
\prompt{In}{incolor}{59}{\boxspacing}
\begin{Verbatim}[commandchars=\\\{\}]
\PY{c+c1}{\PYZsh{} Create a new model with inputs specified by base\PYZus{}inputs dict}
\PY{n}{model\PYZus{}6} \PY{o}{=} \PY{n}{BookstoreModel}\PY{p}{(}\PY{o}{*}\PY{o}{*}\PY{n}{base\PYZus{}inputs}\PY{p}{)}
\PY{n+nb}{print}\PY{p}{(}\PY{n}{model\PYZus{}6}\PY{p}{)}
\PY{n}{model\PYZus{}6}\PY{o}{.}\PY{n}{profit}\PY{p}{(}\PY{p}{)}
\end{Verbatim}
\end{tcolorbox}

    \begin{Verbatim}[commandchars=\\\{\}]
\{'unit\_cost': 7.5, 'selling\_price': 10.0, 'unit\_refund': 2.5, 'order\_quantity':
200, 'demand': 193\}
    \end{Verbatim}

            \begin{tcolorbox}[breakable, size=fbox, boxrule=.5pt, pad at break*=1mm, opacityfill=0]
\prompt{Out}{outcolor}{59}{\boxspacing}
\begin{Verbatim}[commandchars=\\\{\}]
447.5
\end{Verbatim}
\end{tcolorbox}
        
    \begin{tcolorbox}[breakable, size=fbox, boxrule=1pt, pad at break*=1mm,colback=cellbackground, colframe=cellborder]
\prompt{In}{incolor}{60}{\boxspacing}
\begin{Verbatim}[commandchars=\\\{\}]
\PY{c+c1}{\PYZsh{} Specify input ranges for scenarios (dictionary)}
\PY{c+c1}{\PYZsh{} 1\PYZhy{}way table}
\PY{n}{dt\PYZus{}param\PYZus{}ranges\PYZus{}1} \PY{o}{=} \PY{p}{\PYZob{}}\PY{l+s+s1}{\PYZsq{}}\PY{l+s+s1}{demand}\PY{l+s+s1}{\PYZsq{}}\PY{p}{:} \PY{n}{np}\PY{o}{.}\PY{n}{arange}\PY{p}{(}\PY{l+m+mi}{70}\PY{p}{,} \PY{l+m+mi}{321}\PY{p}{,} \PY{l+m+mi}{25}\PY{p}{)}\PY{p}{\PYZcb{}}

\PY{c+c1}{\PYZsh{} 2\PYZhy{}way table}
\PY{n}{dt\PYZus{}param\PYZus{}ranges\PYZus{}2} \PY{o}{=} \PY{p}{\PYZob{}}\PY{l+s+s1}{\PYZsq{}}\PY{l+s+s1}{demand}\PY{l+s+s1}{\PYZsq{}}\PY{p}{:} \PY{n}{np}\PY{o}{.}\PY{n}{arange}\PY{p}{(}\PY{l+m+mi}{70}\PY{p}{,} \PY{l+m+mi}{321}\PY{p}{,} \PY{l+m+mi}{25}\PY{p}{)}\PY{p}{,}
                     \PY{l+s+s1}{\PYZsq{}}\PY{l+s+s1}{order\PYZus{}quantity}\PY{l+s+s1}{\PYZsq{}}\PY{p}{:} \PY{n}{np}\PY{o}{.}\PY{n}{arange}\PY{p}{(}\PY{l+m+mi}{70}\PY{p}{,} \PY{l+m+mi}{321}\PY{p}{,} \PY{l+m+mi}{50}\PY{p}{)}\PY{p}{\PYZcb{}}

\PY{c+c1}{\PYZsh{} Specify desired outputs (list)}
\PY{n}{outputs} \PY{o}{=} \PY{p}{[}\PY{l+s+s1}{\PYZsq{}}\PY{l+s+s1}{profit}\PY{l+s+s1}{\PYZsq{}}\PY{p}{,} \PY{l+s+s1}{\PYZsq{}}\PY{l+s+s1}{order\PYZus{}cost}\PY{l+s+s1}{\PYZsq{}}\PY{p}{]}

\PY{c+c1}{\PYZsh{} Use data\PYZus{}table function to create 1\PYZhy{}way data table}
\PY{n}{m6\PYZus{}dt1\PYZus{}df} \PY{o}{=} \PY{n}{data\PYZus{}table}\PY{p}{(}\PY{n}{model\PYZus{}6}\PY{p}{,} \PY{n}{dt\PYZus{}param\PYZus{}ranges\PYZus{}1}\PY{p}{,} \PY{n}{outputs}\PY{p}{)}
\PY{n}{m6\PYZus{}dt1\PYZus{}df}
\end{Verbatim}
\end{tcolorbox}

            \begin{tcolorbox}[breakable, size=fbox, boxrule=.5pt, pad at break*=1mm, opacityfill=0]
\prompt{Out}{outcolor}{60}{\boxspacing}
\begin{Verbatim}[commandchars=\\\{\}]
    demand  profit  order\_cost
0       70  -475.0      1500.0
1       95  -287.5      1500.0
2      120  -100.0      1500.0
3      145    87.5      1500.0
4      170   275.0      1500.0
5      195   462.5      1500.0
6      220   500.0      1500.0
7      245   500.0      1500.0
8      270   500.0      1500.0
9      295   500.0      1500.0
10     320   500.0      1500.0
\end{Verbatim}
\end{tcolorbox}
        
    \begin{tcolorbox}[breakable, size=fbox, boxrule=1pt, pad at break*=1mm,colback=cellbackground, colframe=cellborder]
\prompt{In}{incolor}{61}{\boxspacing}
\begin{Verbatim}[commandchars=\\\{\}]
\PY{c+c1}{\PYZsh{} Use data\PYZus{}table function to create 2\PYZhy{}way data table}
\PY{n}{m6\PYZus{}dt2\PYZus{}df} \PY{o}{=} \PY{n}{data\PYZus{}table}\PY{p}{(}\PY{n}{model\PYZus{}6}\PY{p}{,} \PY{n}{dt\PYZus{}param\PYZus{}ranges\PYZus{}2}\PY{p}{,} \PY{n}{outputs}\PY{p}{)}
\PY{n}{m6\PYZus{}dt2\PYZus{}df}
\end{Verbatim}
\end{tcolorbox}

            \begin{tcolorbox}[breakable, size=fbox, boxrule=.5pt, pad at break*=1mm, opacityfill=0]
\prompt{Out}{outcolor}{61}{\boxspacing}
\begin{Verbatim}[commandchars=\\\{\}]
    demand  order\_quantity  profit  order\_cost
0       70              70   175.0       525.0
1       70             120   -75.0       900.0
2       70             170  -325.0      1275.0
3       70             220  -575.0      1650.0
4       70             270  -825.0      2025.0
..     {\ldots}             {\ldots}     {\ldots}         {\ldots}
61     320             120   300.0       900.0
62     320             170   425.0      1275.0
63     320             220   550.0      1650.0
64     320             270   675.0      2025.0
65     320             320   800.0      2400.0

[66 rows x 4 columns]
\end{Verbatim}
\end{tcolorbox}
        
    Let's plot the 2-way results using Seaborn.

    \begin{tcolorbox}[breakable, size=fbox, boxrule=1pt, pad at break*=1mm,colback=cellbackground, colframe=cellborder]
\prompt{In}{incolor}{62}{\boxspacing}
\begin{Verbatim}[commandchars=\\\{\}]
\PY{n}{profit\PYZus{}dt\PYZus{}g} \PY{o}{=} \PY{n}{sns}\PY{o}{.}\PY{n}{FacetGrid}\PY{p}{(}\PY{n}{m6\PYZus{}dt2\PYZus{}df}\PY{p}{,} \PY{n}{col}\PY{o}{=}\PY{l+s+s2}{\PYZdq{}}\PY{l+s+s2}{order\PYZus{}quantity}\PY{l+s+s2}{\PYZdq{}}\PY{p}{,} \PY{n}{sharey}\PY{o}{=}\PY{k+kc}{True}\PY{p}{,} \PY{n}{col\PYZus{}wrap}\PY{o}{=}\PY{l+m+mi}{3}\PY{p}{)}
\PY{n}{profit\PYZus{}dt\PYZus{}g} \PY{o}{=} \PY{n}{profit\PYZus{}dt\PYZus{}g}\PY{o}{.}\PY{n}{map}\PY{p}{(}\PY{n}{plt}\PY{o}{.}\PY{n}{plot}\PY{p}{,} \PY{l+s+s2}{\PYZdq{}}\PY{l+s+s2}{demand}\PY{l+s+s2}{\PYZdq{}}\PY{p}{,} \PY{l+s+s2}{\PYZdq{}}\PY{l+s+s2}{profit}\PY{l+s+s2}{\PYZdq{}}\PY{p}{)}
\end{Verbatim}
\end{tcolorbox}

    \begin{center}
    \adjustimage{max size={0.9\linewidth}{0.9\paperheight}}{output_164_0.png}
    \end{center}
    { \hspace*{\fill} \\}
    
    \begin{tcolorbox}[breakable, size=fbox, boxrule=1pt, pad at break*=1mm,colback=cellbackground, colframe=cellborder]
\prompt{In}{incolor}{63}{\boxspacing}
\begin{Verbatim}[commandchars=\\\{\}]
\PY{n}{profit\PYZus{}dt\PYZus{}g}\PY{o}{.}\PY{n}{savefig}\PY{p}{(}\PY{l+s+s1}{\PYZsq{}}\PY{l+s+s1}{images/two\PYZus{}way\PYZus{}dt.png}\PY{l+s+s1}{\PYZsq{}}\PY{p}{)}
\end{Verbatim}
\end{tcolorbox}

    Perfect! The plots make sense in that the order quantity acts as an
upper limit on profit since we end up with unmet demand whenever demand
exceeds our order quantity.

    \hypertarget{concluding-thoughts-and-next-steps}{%
\subsection{Concluding thoughts and next
steps}\label{concluding-thoughts-and-next-steps}}

We've created an OO version of a simple spreadsheet type model that
lends itself pretty well to the type of sensitivity analysis we would do
with something like Excel's Data Table tool. Our Python
\texttt{data\_table} function actually allows \emph{n} input variables
and \emph{m} output variables. In Excel, you can either do a \(1xn\) (a
one-way Data Table) or \(2x1\) (a two-way Data Table).

Along the way we learned the basics of doing OOP in Python and a bunch
of more advanced Python functions and techniques. While that might have
seemed like a bunch of work, now we've got a reusable model object and
reusable data table function that we can build on.

In the next post, we'll build on this work and take on implementing an
Excel-style Goal Seek function in Python. We'll explore this using both
the OO and the non-OO models. Each leads to some interesting challenges.
Which approach do you think will end up being easier to use within a
Python \texttt{goal\_seek} function?

Then we'll use these same models for Monte-Carlo simulation. We'll be
creating our own \texttt{simulate} function to add to the
\texttt{data\_table} and \texttt{goal\_seek} functions that we created
in the first and second parts of this series.

In the fourth post in this series, we'll learn to create a Python
package containing our three ``what-if?'' functions so that we can share
our work and easily use these functions in different notebooks or Python
programs.

    \hypertarget{answers}{%
\subsection{ANSWERS}\label{answers}}

    \textbf{QUESTION 1:} Complete the lines of code below (answer at bottom
of notebook)

    \begin{tcolorbox}[breakable, size=fbox, boxrule=1pt, pad at break*=1mm,colback=cellbackground, colframe=cellborder]
\prompt{In}{incolor}{ }{\boxspacing}
\begin{Verbatim}[commandchars=\\\{\}]
\PY{n}{order\PYZus{}cost} \PY{o}{=} \PY{n}{unit\PYZus{}cost} \PY{o}{*} \PY{n}{order\PYZus{}quantity}
\PY{n}{sales\PYZus{}revenue} \PY{o}{=} \PY{n+nb}{min}\PY{p}{(}\PY{n}{order\PYZus{}quantity}\PY{p}{,} \PY{n}{demand}\PY{p}{)} \PY{o}{*} \PY{n}{selling\PYZus{}price}
\PY{n}{refund\PYZus{}revenue} \PY{o}{=} \PY{n+nb}{max}\PY{p}{(}\PY{l+m+mi}{0}\PY{p}{,} \PY{n}{order\PYZus{}quantity} \PY{o}{\PYZhy{}} \PY{n}{demand}\PY{p}{)} \PY{o}{*} \PY{n}{unit\PYZus{}refund}
\PY{n}{profit} \PY{o}{=} \PY{n}{sales\PYZus{}revenue} \PY{o}{+} \PY{n}{refund\PYZus{}revenue} \PY{o}{\PYZhy{}} \PY{n}{order\PYZus{}cost}
\end{Verbatim}
\end{tcolorbox}

    \textbf{CHALLENGE 2} Lost sales method

We could add the following method to our \texttt{BookstoreModel} class.

    \begin{tcolorbox}[breakable, size=fbox, boxrule=1pt, pad at break*=1mm,colback=cellbackground, colframe=cellborder]
\prompt{In}{incolor}{ }{\boxspacing}
\begin{Verbatim}[commandchars=\\\{\}]
\PY{k}{def} \PY{n+nf}{lost\PYZus{}sales}\PY{p}{(}\PY{n+nb+bp}{self}\PY{p}{)}\PY{p}{:}
    \PY{l+s+sd}{\PYZdq{}\PYZdq{}\PYZdq{}Compute number of items demanded but no inventory to sell\PYZdq{}\PYZdq{}\PYZdq{}}
    \PY{k}{return} \PY{n}{np}\PY{o}{.}\PY{n}{maximum}\PY{p}{(}\PY{l+m+mi}{0}\PY{p}{,} \PY{n+nb+bp}{self}\PY{o}{.}\PY{n}{demand} \PY{o}{\PYZhy{}} \PY{n+nb+bp}{self}\PY{o}{.}\PY{n}{order\PYZus{}quantity}\PY{p}{)}
\end{Verbatim}
\end{tcolorbox}

    \begin{tcolorbox}[breakable, size=fbox, boxrule=1pt, pad at break*=1mm,colback=cellbackground, colframe=cellborder]
\prompt{In}{incolor}{ }{\boxspacing}
\begin{Verbatim}[commandchars=\\\{\}]

\end{Verbatim}
\end{tcolorbox}


    % Add a bibliography block to the postdoc
    
    
    
\end{document}
